\subsection{Pottsmodel}
It can be considered as generalisation of the Ising model, where the spin can take on more than just two values. Usually they are enumerated $s\in \{1,2,\ldots,q\}$. It is called the {\em q-state Potts model}. The   Hamiltonian of the Potts model reads
%
\tboxitp{Pottsmodel}{$q$ states}{
\begin{align}
H_{P} &= -J \sum_{ij} \delta_{s_{i}s_{j}}\;,
\end{align}}
%
i.e. when spins on neighbouring sites are equal, they experience a lower energy ($E=-J$)
than when the have different spin values ($E=0$).
The Potts model is used to study  the behaviour of ferromagnets and other
phenomena of solid-state physics. 

Apart from statistical physics it is also used in computer science (signal processing) and biology (neural networks). 

The strength of the Potts model is not so much 
that it models these physical systems well; it is rather that the one-dimensional case is exactly 
solvable, and that it has interesting physical properties. For $d\ge 2$ it shows a phase transition.
It is second order for $q\le 4$ and first order for $q>4$.






\subsection{Maxwell Relations}

We know
%
\begin{align*}
d U(S,V,N) &= T dS -p dV +\mu dN
\end{align*}
%
Hence
%

\begin{align*}
\pder{U(S,V,N)}{S}{V,N} &= T(S,V,N)\\
\pder{U(S,V,N)}{V}{S,N} &= -p(S,V,N)\;.
\end{align*}
%
From this we obtain
%
\begin{align*}
\pder{T}{V}{S,N} &=\frac{\partial }{\partial V}\bigg(\pder{U}{S}{V,N}\bigg)\at_{S,N}
 =\frac{\partial }{\partial S}\bigg(\pder{U}{V}{S,N}\bigg)\at_{V,N}
 =-\pder{p}{S}{V,N}
\end{align*}
%
In the following we use the convention
%
\begin{align*}
\frac{\partial^{2}f(x,y,z)}{\partial x\partial y } 
=\frac{\partial }{\partial y}\bigg(\pder{f}{x}{y,z}\bigg)\at_{x,z}\;.
\end{align*}
%
and we can simplify the former equation to
%
\begin{align*}
\pder{T}{V}{S,N} &= -\pder{p}{S}{V,N}
 =\frac{\partial^{2}U(S,V,N) }{\partial S\partial V}\;.
\end{align*}
%
Similarly we find with
\begin{align*}
\pder{F(T,V,N)}{T}{V,N} &= -S(T,V ,N)\\
\pder{F(T,V,N)}{V}{S,N} &= -p(T,V ,N)\;.
\end{align*}
%
the relation
%
\begin{align*}
\pder{S(T,V,N)}{V}{T,N} & =\pder{p(T,V,N)}{T}{V,N}
=-\frac{\partial^{2} F(T,V,N)}{\partial V\partial T}
\end{align*}
%
Based on the free enthalpy $G(T,p,N)$
we obtain with
%
\begin{align*}
\pder{G(T,p,N)}{T}{p,N} &= -S(T,p ,N)\\
\pder{G(T,p,N)}{p}{S,N} &= V(T,p ,N)\;.
\end{align*}
%
the relation
%
\begin{align*}
-\pder{S(T,p,N)}{p}{T,N} & =\pder{V(T,p,N)}{T}{p,N}
=\frac{\partial^{2} G(T,p,N)}{\partial p\partial T}
\end{align*}
%
Finally, we introduce the {\em enthalpy} $H(S,p,N)$ via the Legendre transformation of
$U(S,V,N)$ with respect to $V$, i.e.
%
\begin{align*}
H(S,p,N) &= U(S,V,N) + pV\;,
\end{align*}
% 
for which we have
%
\begin{align*}
\pder{H(S,p,N)}{S}{p,N} &= T(S,p ,N)\\
\pder{H(S,p,N)}{p}{S,N} &= V(S,p ,N)\;.
\end{align*}
%
Then we find
\begin{align*}
\pder{T(S,p,N)}{p}{S,N} & =\pder{V(S,p,N)}{S}{p,N}
=\frac{\partial^{2} H(S,p,N)}{\partial p\partial S}
\end{align*}

We summaries these results 
\tboxit{Maxwellrelations}{
\begin{subequations}\label{eq:}
\begin{align}
\pder{T(S,V,N)}{V}{S,N} &= -\pder{p(S,V,N)	}{S}{V,N}
 &&=\frac{\partial^{2}U(S,V,N) }{\partial S\partial V}\\
\pder{S(T,V,N)}{V}{T,N} & =\pder{p(T,V,N)}{T}{V,N}
&&=-\frac{\partial^{2} F(T,V,N)}{\partial V\partial T}\\
-\pder{S(T,p,N)}{p}{T,N} & =\pder{V(T,p,N)}{T}{p,N}
&&=\frac{\partial^{2} G(T,p,N)}{\partial p\partial T}\\
\pder{T(S,p,N)}{p}{S,N} & =\pder{V(S,p,N)}{S}{p,N}
&&=\frac{\partial^{2} H(S,p,N)}{\partial p\partial S}\;.
\end{align}
\end{subequations}
}

\subsection{Ehrenfest equations}

Next we derive the analogue of Clausius-Clapeyron for second order phase transitions.
In this case, $S$ and $V$ are continuous when crossing 
the phase boundary. Along a phase boundary between phase $\alpha$ and $\beta$ we therefore have in the Gibbs ensemble $(T,p,N)$ fixed
%
\begin{align*}
S_{\alpha}(T,p,N) &= S_{\beta}(T,p,N)\\
V_{\alpha}(T,p,N) &= V_{\beta}(T,p,N)\;,
\end{align*}
%
and also
%
\begin{align*}
dS_{\alpha} &= dS_{\beta}\\
dV_{\alpha} &= dV_{\beta}\;.
\end{align*}
%
We can use these equations, to obtain $dp/dT$ along the phase boundary.
We combine  these equation as
%
\begin{align*}
dX_{\alpha}(T,p,N) &= dX_{\beta}(T,p,N)\;,
\end{align*}
%
where $X$ stands either for $S$ or $V$.
Since $N$ is fixed, it will not vary along the phase boundary. We therefore have
\begin{align*}
\pder{X_{\alpha}}{T}{p,N} dT +\pder{X_{\alpha}}{p}{T,N} dp
&=\pder{X_{\beta}}{T}{p,N} dT +\pder{X_{\beta}}{p}{T,N} dp\;,
\end{align*}
%
from which we obtain
%
\begin{align*}
\bigg(\pder{X_{\beta}}{T}{p,N}-\pder{X_{\alpha}}{T}{p,N}\bigg)dT
&=
-\bigg(\pder{X_{\beta}}{p}{T,N}-\pder{X_{\alpha}}{p}{T,N}\bigg)dp\\
\frac{dp}{dT}
&=
-\frac{\pder{X_{\beta}}{T}{p,N}-\pder{X_{\alpha}}{T}{p,N}}
{\pder{X_{\beta}}{p}{T,N}-\pder{X_{\alpha}}{p}{T,N}}
:=
-\frac{\Delta \pder{X_{\beta}}{T}{p,N}}
{\pder{\Delta X_{\beta}}{p}{T,N}}
\end{align*}
%
If we insert $S$ for $X$ we find
%
\begin{align}\label{eq:}
\frac{d p}{dT} &=-\frac{\Delta \pder{S}{T}{p,N}}
{\pder{\Delta S}{p}{T,N}}\;.
\end{align}
%
We use 
%
\begin{align*}
\pder{S}{T}{p,N} &= \frac{1}{T} C_{p}\;,
\end{align*}
%
and from the Maxwell relations we use 
%
\begin{align*}
\pder{S}{p}{T,N} &= -\pder{V}{T}{p,N} \;.
\end{align*}
%
The change in volume w.r.t. temperature is related to the 
$\alpha$
%
\tboxitp{coefficient of thermal expansion}{for constant pressure}{
\begin{align}
\alpha_{p} &:= \frac{1}{V}\pder{V}{T}{p,N}  
\end{align}}
%
With these response functions, we find
%
\begin{align}\label{eq:}
\frac{dp}{dT}\at_\text{ph.b.} &= \frac{1}{V T}\frac{\Delta C_{p}}{\Delta \alpha_{p}}\;.
\end{align}
%
Alternatively, if we use $V$ as $X$, we have
%
\begin{align}\label{eq:}
\frac{d p}{dT} &=-\frac{\Delta \pder{V}{T}{p,N}}
{\pder{\Delta V}{p}{T,N}}\;.
\end{align}
%
Here we employ the relations
%
\begin{align*}
\alpha_{p} &= \frac{1}{V} \pder{V}{T}{p,N}\\
\kappa_{p} &= -\frac{1}{V} \pder{V}{p}{T,N} \;,
\end{align*}
%
and find
\begin{align}\label{eq:}
\frac{dp}{dT}\at_\text{ph.b.} &= \frac{\Delta \alpha_{p}}{\Delta \kappa_{p}}\;.
\end{align}

In summary we have the
\tboxit{Ehrenfest equations}{
%
\begin{align}\label{eq:}
\frac{dp}{dT}\at_\text{ph.b.} &= \frac{\Delta \alpha_{p}}{\Delta \kappa_{p}}
=\frac{1}{V T}\frac{\Delta C_{p}}{\Delta \alpha_{p}}\;.
\end{align}
%
}


\section{Ising again}

%
\begin{align*}
s&:=\sinh(2K)\\
f=\frac{F}{JN} &= -K\bigg[\frac{\ln(2)}{2} +\frac{1}{4\pi}\int_{0}^{2\pi}
\log\bigg( 1+ s^{4} + \sqrt{1+s^{4} -2 s^{2}\cos(x)}\bigg) dx\bigg]
\end{align*}
%
$T_{c}$ from
%
\begin{align*}
s_{c} &= \sinh(2K_{c}) = 1\\
K_{c} &= J \beta_{c} = 0.440687\\
\frac{k_{B} T_{c}}{J} &=2.26919\;.
\end{align*}
%
%
\begin{align*}
u=\frac{U}{JN} &= -   \coth(2 K)
\bigg[ 
1+\frac{2( 2 \tanh^{2}\big( 2K \big)-1 )}{\pi}
\int_{0}^{\pi/2}\frac{1}
{
\sqrt{1-\frac{4k}{(1+k)^{2}}\sin^{2}(x)}
}
 \bigg]
\end{align*}
%


For $T\le T_{c}$
%
\begin{align*}
M(T) &= \bigg[ 1 - \bigg(\sinh^{-4}(2K)  \bigg) \bigg]^{1/8}\;.
\end{align*}
%

%
\begin{align*}
S &= -\frac{\partial F}{\partial T} = -\frac{\partial F}{\partial \beta} \frac{d\beta}{dT}
= \frac{J}{k_{B}T^{2}} \frac{\partial F}{\partial J \beta} \\
&= \frac{K}{T} \frac{\partial F}{\partial K} = \frac{K J N}{T} \frac{\partial f}{\partial K}\\
\frac{S}{N}&= k_{B}\frac{K J}{k_{B}T} \frac{\partial f}{\partial K}\\
s&=\frac{S}{k_{B}N}= K^{2} \frac{\partial f}{\partial K}\\
\end{align*}
%
%
\begin{align*}
U &= F+TS = f J N + T s k_{B} N  \\
&= JN\bigg( f  + \frac{T  k_{B}}{J} s\bigg) \\
u&= f + \frac{s}{K}
\end{align*}
%

%
\begin{align*}
s&= K f - K^{3}\frac{d}{dK} \frac{1}{4\pi}\int_{0}^{2\pi}
\log\bigg( 1+s^{2} +  \sqrt{1+s^{4} -2 s^{2}\cos(x)}\bigg) dx\\
&= K f - K^{3}\bigg(\frac{d}{ds} \frac{1}{4\pi}\int_{0}^{2\pi}
\log\bigg( 1+s^{2} +  \sqrt{1+s^{4} -2 s^{2}\cos(x)}\bigg) dx\bigg)
\;\frac{d}{dK} \sinh(2K)\\
&= K \bigg(f - K^{2}\bigg(\frac{d}{ds} \frac{1}{4\pi}\int_{0}^{2\pi}
\log\bigg( 1+s^{2} +  \sqrt{1+s^{4} -2 s^{2}\cos(x)}\bigg) dx\bigg)!
\;2 \sqrt{1+s^{2}}\bigg)\;.
\end{align*}
%

%
\begin{align*}
u &= f + \frac{s}{K}
\end{align*}
%




