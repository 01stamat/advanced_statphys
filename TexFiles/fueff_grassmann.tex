%!TEX root =    a_fueff.tex
%newcommand{\Int}{\int\ldots\int}
\newcommand{\sgn}{{\rm sgn}}
\chapter{Grassmann Algebra}
%****************************************************************************************************************************************************************
\subsection{Matrix elements of $M$, $M b^{\dagger}$, and $M b$ in phonon eigen vectors
 }
 
 \subsection{Bogolubov}
 
 
 \subsubsection{Useful relation for coherent states}

We proof that 
%
\begin{align}
e^{A+B} &= e^{A} e^{B} e^{-\frac{1}{2}[A,B]}\label{eq:Glauber}
\end{align}
%
provided, $[A,B]$ commutes with $A$ and $B$.
\subsubsection{Proof due to Glauber}
%
\begin{align*}
f(x)&:=e^{x A}e^{x B}\\
f'(x) &=e^{x A}A e^{x B} +e^{x A}e^{x B} B\\
&=e^{x A} e^{x B}  e^{-x B}A e^{x B} +e^{x A}e^{x B} B\\
&=f(x) \bigg(  \underbrace{e^{-x B}A e^{x B}}_{:=A(x)} + B\bigg)\;.
\end{align*}
%
%
\begin{align*}
A'(x) &= e^{-x B}[A,B] e^{x B} \overset{[[A,B],B]=0}{=} [A,B]\\
A(x) &= A + [A,B] \;x
\end{align*}
%
%
\begin{align*}
f'(x) &=f(x) \bigg(\big(A+B\big) + [A,B] \;x\bigg)\\
f(x) &= e^{\big(A+B\big)x} e^{\frac{1}{2}[A,B]x^{2}}\\
f(1)&=  e^{A+B} e^{\frac{1}{2}[A,B]}\\
&\overset{\text{def.}}{=} e^{A}e^{B}
\end{align*}
%
\subsubsection{Application to coherent states}
We consider
%
\begin{align*}
U(\gamma)&:= e^{\overbrace{-\gamma\dag{a}{}}^{:=A}+\overbrace{\gamma^{*}\nodag{a}{}}^{:=B}} \;.
\end{align*}
The commutator $[A,B]$ is
%
\begin{align*}
[A,B] &= -\abs{\gamma}^{2} [\dag{a}{},\nodag{a}{}] =  \abs{\gamma}^{2}\;.
\end{align*}
%
Clearly, it commutes with all operators. So we can use the above equation
and obtain
%
\begin{align}\label{eq:coherent:1}
 e^{-\gamma\dag{a}{}+\gamma^{*}\nodag{a}{}} &= 
 e^{-\frac{\abs{\gamma}^{2}}{2}}\;
 e^{-\gamma \dag{a}{}} e^{\gamma^{*} \nodag{a}{}} 
\end{align}
%

Alternatively, we can start from
%
\begin{align*}
U(\gamma)&:= e^{\overbrace{\gamma^{*}\nodag{a}{}}^{:=A}\overbrace{-\gamma\dag{a}{}}^{:=B}} \;.
\end{align*}
The commutator $[A,B]$ is
%
\begin{align*}
[A,B] &= \abs{\gamma}^{2} [\nodag{a}{},\dag{a}{}] =  -\abs{\gamma}^{2}\;.
\end{align*}
%
And we end up with
%
\begin{align*}
 e^{\gamma^{*}\nodag{a}{}-\gamma\dag{a}{}} &= 
 e^{\frac{\abs{\gamma}^{2}}{2}}\;
e^{\gamma^{*} \nodag{a}{}} \; e^{-\gamma \dag{a}{}} \;.
\end{align*}
%
We use \eq{eq:coherent:1} to compute the following matrix elements
%
\begin{align*}
\bra{m} e^{\gamma^{*}\nodag{a}{}-\gamma\dag{a}{}} \ket{n} &=
e^{-\frac{1}{2}\abs{\gamma}^{2}} 
\bra{m} e^{-\gamma\dag{a}{}} 
e^{\gamma^{*}\nodag{a}{}}
\ket{n}
\end{align*}
%
%
\begin{align*}
e^{\gamma^{*}\nodag{a}{}} \ket{n} &= 
\sum_{\nu=0}^{\infty} \frac{(\gamma^{*})^{\nu}}{\nu!}
(\nodag{a}{})^{\nu} \frac{1}{\sqrt{n!}} (\dag{a}{})^{n}\ket{0}
\end{align*}
%
Clearly, the sum is restricted to $\nu\le n$ end we get
\begin{align*}
e^{\gamma^{*}\nodag{a}{}} \ket{n} &= 
\sum_{\nu=0}^{n} \frac{(\gamma^{*})^{\nu}}{\nu!}
\frac{\sqrt{(n-\nu)!}}{\sqrt{n!}} \ket{n-\nu}
\end{align*}
Similarly,
\begin{align*}
e^{-\gamma^{*}\nodag{a}{}} \ket{m} &= 
\sum_{\mu=0}^{m} \frac{(-\gamma^{*})^{\mu}}{\mu!}
\frac{\sqrt{(m-\mu)!}}{\sqrt{m!}} \ket{m-\mu}\\
\bra{m} e^{-\gamma\dag{a}{}}  &= 
\sum_{\mu=0}^{m} \frac{(-\gamma)^{\mu}}{\mu!}
\frac{\sqrt{(m-\mu)!}}{\sqrt{m!}} \bra{m-\mu}\;.
\end{align*}

The sought-for matrix element therefore reads according to \eq{eq:coherent:1}
%
\begin{align*}
\bra{m} e^{\gamma^{*}\nodag{a}{}-\gamma\dag{a}{}} \ket{n} &=
e^{-\frac{1}{2}\abs{\gamma}^{2}}
\sum_{\mu=0}^{m} \frac{(-\gamma)^{\mu}}{\mu!}
\frac{\sqrt{(m-\mu)!}}{\sqrt{m!}} 
%%
\sum_{\nu=0}^{n} \frac{(\gamma^{*})^{\nu}}{\nu!}
\frac{\sqrt{(n-\nu)!}}{\sqrt{n!}} \underbrace{\braket{m-\mu}{n-\nu}}_{=\delta_{m-\mu,n-\nu}}
\end{align*}
%
For $m\ge n$ we eliminate $\mu$ with $\delta_{\mu,m-n+\nu}$. The
constraint due the sum $\sum_{\mu=0}^{m}$ is $0\le \mu\le m$ and it yields
%
\begin{align*}
0&\le m-n+\nu \le m\\
-(m-n)&\le \nu\qquad \wedge \nu\le n
\end{align*}
Both conditions are fulfilled by the constraint of the sum over $\nu$. So we get
%
\begin{align*}
\bra{m} e^{\gamma^{*}\nodag{a}{}-\gamma\dag{a}{}} \ket{n} &=
e^{-\frac{1}{2}\abs{\gamma}^{2}}
\sum_{\nu=0}^{n}  \frac{(-\gamma)^{m-n+\nu}}{(m-n+\nu)!}
\frac{\sqrt{(m-(m-n+\nu))!}}{\sqrt{m!}} 
%%
 \frac{(\gamma^{*})^{\nu}}{\nu!}
\frac{\sqrt{(n-\nu)!}}{\sqrt{n!}} \\
&=e^{-\frac{1}{2}\abs{\gamma}^{2}}
\frac{(-\gamma)^{m-n}}{\sqrt{m!n!}}
\sum_{\nu=0}^{n} \big(-\abs{\gamma}^{2}\big)^{\nu}\;
%%
 \frac{(n-\nu)! 
 }{\nu!(m-n+\nu)!}
\;:
\end{align*}
For $m\le n$ we eliminate $\nu$ with $\delta_{\nu,n-m+\mu}$. The
constraint due the sum $\sum_{\nu=0}^{n}$ is $0\le \nu\le n$ and it yields
%
\begin{align*}
0&\le n-m+\mu \le n\\
-(n-m)&\le \mu\qquad \wedge \mu\le m
\end{align*}
Both conditions are again fulfilled by the constraint of the sum over $\mu$. 
We obtain
%
\begin{align*}
\bra{m} e^{\gamma^{*}\nodag{a}{}-\gamma\dag{a}{}} \ket{n} 
&=e^{-\frac{1}{2}\abs{\gamma}^{2}}\frac{1}{\sqrt{n! m!}}
\sum_{\mu=0}^{m} \frac{(-\gamma)^{\mu}}{\mu!}
\sqrt{(m-\mu)!}
%%
\frac{(\gamma^{*})^{n-m+\mu}}{(n-m+\mu)!}
\sqrt{(n-(n-m+\mu))!}\\
&=e^{-\frac{1}{2}\abs{\gamma}^{2}}\frac{(\gamma^{*})^{n-m}}{\sqrt{n! m!}}\;
\sum_{\mu=0}^{m} \big(-\abs{\gamma}^{2}\big)^{\mu}
%%
\frac{(m-\mu)!}{\mu!(n-m+\mu)!}
\end{align*}
%
So in summary we have
\tboxit{}{
%
\begin{align*}
\bra{m} e^{\gamma^{*}\nodag{a}{}-\gamma\dag{a}{}} \ket{n} 
&=e^{-\frac{1}{2}\abs{\gamma}^{2}}\frac{c^{|n-m|}}{\sqrt{n! m!}}\;
\sum_{\mu=0}^{L} \big(-\abs{\gamma}^{2}\big)^{\mu}
%%
\frac{(L-\mu)!}{\mu!\;(|n-m|+\mu)!}\\
L &=\min(m,n)\\
c&=
\begin{cases}
-\gamma &\text{for } n\le m\\
\gamma^{*} &\text{for } m\le n
\end{cases}
\end{align*}
%
}

 
%****************************************************************************************************************************************************************
\section {Das Dimer Problem}
%****************************************************************************************************************************************************************

Wir wollen alle erlaubten Dimer-Konfigurationen über Grassmann Variablen erzeugen.
Wir betrachten dazu die Exponentialfunktion
\begin{align*}
    \omega(\xi) &= e^{\sum_{i<j} b_{ij} \xi_i\xi_j}\;.
\end{align*}
Hierbei läuft $i$ über die $L_x$ x-Koordinaten der Gitterpunkte und $j$ entsprechend über die 
$L_y$ y-Koordinaten der Gitterpunkte. Insgesamt gibt es demnach  $N=L_x*L_y$ Gitterpunkt. Die Zahl der Dimere ist $N_D=N/2$. Eine vollständige Dimer-Belegung ist
natürlich nur möglich, wenn die Zahl der Plätze gerade ist.
Zunächst werden wir alle $b_{ij}$, die nicht zu nächsten Nachbar Paaren gehören auf Null setzen. Damit kommen in den Summen nur noch Indizes zu 
benachbarten Gitterplätzen vor.
Da Paare von Grassmann Variablen  zu ${\cal A}_+$ gehören, vertauschen sie. Außerdem ist verschwindet das Quadrat und jede höhere 
Potenz dieser Paare und wir können für die Exponentialfunktion umformen in
schreiben
\begin{align*}
    \omega(\xi) &= \prod_{i<j}\; e^{b_{ij} \xi_i\xi_j}  \\
    &= \prod_{i<j}\;\bigg(1 + b_{ij}\xi_i\xi_j    \bigg)\;,
\end{align*}
Jeder Faktor steht für genau eines der $N_{nn}=2*N$ n.N. Paare.
Wenn wir die Produkte ausrechnen, erhalten 
wir eine Summe, bei der das Paar $8ij)$ entweder als $1=:(b_{ij} \xi_i\xi_j)^0$ oder $b_{ij} \xi_i\xi_j=:(b_{ij} \xi_i\xi_j)^1$ beiträgt. Wir führen
pro Paar die "Dimerbesetzung" $n_{ij}=0 (1)$ ein und können die Summe, die aus $2^{N_nn}$ Termen besteht schreiben als
\begin{align}\label{eq:gv1}
    \omega(\xi) &= \sum_{\{n_{ij} \}}\; \prod_{ij}\;(b_{ij}\xi_i\xi_j)^n_{ij}\;.
\end{align}
In dieser Summe dürfen keine Grassmann Variablen wegen des Pauliprinzips doppelt vorkommen.
Wenn wir nun noch über $\omega(\xi)$ integrieren
\begin{align*}
    I &=\Int\;\omega(\xi)\;\prod_{\nu=1}^N d\xi_\nu\;
\end{align*}
tragen nur noch solche Terme der Summe in \eqref{eq:gv1} bei, bei denen jeder Gitterpunkt (und somit jedes $\xi_i$) genau einmal vorkommt. Damit repräsentiert
jeder Term der Summe eine erlaubte Dimer Konfiguration. Die Summe läuft demnach über aller erlaubten Dimer Konfigurationen ${\cal D}$. Die Indizes des $\nu$-ten Paares 
in der der Dimer Konfiguration ${\cal D}$ bezeichnen wir mit $(d^\nu_1,d^\nu_2)$. Aus dem Integral wird dann

\begin{align}\label{eq:gv3}
    I =  \sum_{\cal D}\; \prod_{\nu=1}^{N/2}\;b_{d^\nu_1,d^\nu_2} \Int \; \prod_{\nu=1}^{N/2}\;\xi_{d^\nu_1}\xi_{d^\nu_2}\;\prod_{\nu=1}^N d\xi_\nu\;.
\end{align}
Wir bringen die Grassmann Variablen im Integranden in die absteigende Reihenfolge 
\begin{align}\label{eq:gv2}
    \prod_{\nu=1}^{N/2}\;\xi_{d^\nu_1}\xi_{d^\nu_2} &= 
    \sgn\big(D^\nu\big)\;\prod_{j=N}^1\;\xi_j\;,
\end{align}
wobei $\sgn(D^\nu)$ das Vorzeichen angibt, dass beim Vertauschen der Grassmann Variablen entsteht. Das Integral ergibt nach der Umsortierung der
$\xi$ Eins. Damit haben wir

\begin{align*}
    I =  \sum_{\cal D}\; \underbrace{\prod_{\nu}\;b_{d^\nu_1,d^\nu_2} \;\sgn\big(D^\nu\big)}_{=:\omega(D)}
\end{align*}
Wir betrachten einmal die Dimer Konfiguration $D_0$ mit perfekter paralleler Ausrichtung aller Dimere in x-Richtung. Wenn wir die Gitterplätze in x-Richtung fortlaufend
durchnummerieren ($1,\ldots L_x$ in der  Zeile, $L_x+1,\ldots 2*L_x$, etc.) so haben die Paare immer benachbarte Indizes $(i,i+1)$. Damit erhalten wir wir für diese
Konfiguration den Beitrag
\begin{align*}
    b_{1,2} b_{3,4}\ldots b_{N-1,N} \prod_{i=1}^N \xi_i &=
    b_{1,2} b_{3,4}\ldots b_{N-1,N} \prod_{i=N}^1 \xi_i\;.
\end{align*}
Das Vorzeichen $\sgn(D_0)=1$. Bei den nN Paaren handelt es sich ausschließlich um Paare in x-Richtung. Wir wählen speziell
\begin{equation}\label{eq:gv_b}
    b_{(i_x,i_y),(i_x+1,i_y)} = 1\;,
\end{equation}
dann ist auch das Produkt der b-s Eins und somit auch das Gewicht $\omega(D_0)$ der Konfiguration $D_0$
\begin{equation}\label{eq:gv_d0}
    \omega(D_0) = 1\;.
\end{equation}
Tatsächlich liefert jeder einzelne Dimer individuell den Beitrag $+1$. Hierauf werden wir später zurückkommen und $D_0$ als Referenz Konfiguration auswählen.

Wenn wir erreichen wollen, dass $I=N({\cal D})$, die Zahl der Dimer Konfigurationen ist, dann müssen alle Summanden durch geeignete Wahl der $b_{ij}$
zu Eins gemacht werden. Wir können uns hierbei auf $b_{ij}=\pm 1$ beschränken, da das bereits zum Ziel führen wird.
Jeder der Summanden kann dann auch nur $\pm 1$ sein. Wir vergleichen das Vorzeichen zweier beliebiger Dimer Konfigurationen ${\cal D}$ und ${\cal D}'$.
Legt man diese Konfigurationen im Graphen übereinander, dann entstehen geschlossene Schleifen. Die kleinste Schleife besteht dabei aus einem n.N.-Paar.
Damit das graphisch eine Schleife wird, biegen wir ein Dimer etwas nach oben und das ander etwas nach unten.

Um Paare von Indizes auch in Absteigender Reihenfolge durchlaufen zu können führen wir ein, dass
\begin{align*}
    b_{ji} = - b_{ij}
\end{align*}
sein soll. Damit ist 
\begin{align*}
    b_{ij}\xi_i\xi_j = b_{ji}\xi_j\xi_i\;.
\end{align*}
Wir werden nun alle oben besprochenen Schleifen im Urzeigersinn durchlaufen und die darin vorkommenden Indizes fortlaufend durchnummerieren.
Dabei werden alternierend Dimere von der einen und der anderen Konfiguration enthalten sein. Wir beginnen o.B.d.A. in allen Schleifen mit der Konfiguration ${\cal D}'$.
Alternativ könnten wir auch immer mit ${\cal D}'$ beginnen. In der $\mu$-ten Schleife sind die Indizes 
der darin enthaltenen Dimere von ${\cal D}$ 
\begin{align*}
    (I^\mu_1,I^\mu_2), (I^\mu_3,I^\mu_4),\ldots,(I^\mu_{n^\mu-1,n^\mu})\;,
\end{align*}
die von der Konfiguration ${\cal D}'$ sind dann
\begin{align*}
    (I^\mu_2,I^\mu_3), (I^\mu_4,I^\mu_5),\ldots,(I^\mu_{n^\mu,1})\;,
\end{align*}

Wir multiplizieren nun die Gewichte dieser beiden Dimer Konfigurationen miteinander und faktorisieren sie in die auftretenden Schleifen
\begin{align*}
    \omega({\cal D})\;\omega({\cal D}') &= \prod_{\mu} \omega(S_\nu) \;
\end{align*}
Das erste Produkt läuft über die individuellen Schleifen und $\omega(S_\mu)$ ist das zugehörige Gewicht für das gilt:
\begin{align*}
    \omega(S_\nu)  &=  \prod_{j=1}^{n^\mu/2}\; b_{I^\mu_{2j-1},I^\mu_{2j}} 
\end{align*}
Die Länge der $\mu$-ten Schleife wurde mit $n^\mu$ bezeichnet.
Da jede Schleife eine gerade Anzahl von Punkten enthält, kann man die $\xi$, so wie sie in den Schleifen vorkommen ohne zusätzliche Vorzeichen umsortieren.
Wenn sich in einem Dimer die Reihenfolge ändert, wird das Vorzeichen durch die Antisymmetrie der b-s, die ja ebenfalls umnummeriert werden, abgefangen.

\section{Wahl der b-s}
Wir gehen nun zu \eqref{eq:gv3} zurück und betrachten hiervon den Beitrag der Dimer Konfigurationen ${\cal D}$ und ${\cal D}'$
Wir faktorisieren nach den $N_s^\nu$ Schleifen,
die durch ${\cal D}$ und  ${\cal D}'$ festgelegt werden. Für ${\cal D}$ lautet der Beitrag
\begin{align*}
\prod_{\nu=1}^{N/2}\;b_{d^\nu_1,d^\nu_2}  \; \prod_{\nu=1}^{N/2}\;\xi_{d^\nu_1}\xi_{d^\nu_2} &=
\prod_{\mu=1}^{N^\nu_s}\;
X_\mu^\nu\;.
\end{align*}
%
Analog gilt für ${\cal D}'$
\begin{align*}
\prod_{\nu=1}^{N/2}\;b_{{d'}^\nu_1,{d'}^\nu_2}  \; \prod_{\nu=1}^{N/2}\;\xi_{{d'}^\nu_1}\xi_{{d'}^\nu_2} &=
\prod_{\mu=1}^{N^\nu_s}\;
{X'}_\mu^\nu\;.
\end{align*}
Um die Darstellung einigermaßen transparent zu halten unterdrücken wir im folgenden den Index $\nu$ der zugrunde liegenden Dimerkonfiguration
und den Index $\mu$ der betrachteten Schleife. Für die
Faktoren $X_\mu^\nu$ gilt dann
\begin{align*}
    X = \underbrace{b_{I_1 I_2}\ldots b_{I_{L-1}I_L}}_{X_b}\;\;\xi_{I_1}\ldots\xi_{I_L}\;,
\end{align*}
wobei $L$ die Länge der Schleife ist.
Dieselbe Schleife führt bei der Konfiguration ${\cal D}'$ zu dem Beitrag
\begin{align*}
    X' &= b_{I_2 I_3}\ldots b_{I_{L}I_1}\;\;\xi_{I_2}\ldots\xi_{I_L}\xi_{1}\\
    &= - \underbrace{b_{I_2 I_3}\ldots b_{I_{L}I_1}}_{{X'}_b}\;\;\xi_{I_1}\ldots\xi_{I_L}\\
\end{align*}
Das heißt, die Integration über die GV, die in der Schleife enthalten sind liefert nun in beiden Fällen dasselbe Vorzeichen.
Wenn es uns gelingt, sicherzustellen, dass b-abhängigen Anteile in $X$ und $X'$ unterschiedliche Vorzeichen haben, wäre der Beitrag beider Schleifen 
gleich .




Wir multiplizieren hierzu die b-abhängigen Anteile und gruppieren die Faktoren geeignet um. Die Forderung ist dann
\begin{align*} 
    X_b \;{X'}_b &= \prod_{i=1}^{L-1} b_{I_i,I_{i+1}} \overset{!}{=} -1\;.
\end{align*}
Wenn das durch geeignete Wahl der b-s erreicht werden kann, sind die Beiträge zum Gewicht $\omega(D)$ bzw. $\omega(D')$ in jeder Schleife für $D$ und $D'$
gleich. Wählen wir als Referenz Konfiguration $D=D_0$, dann wissen wir, dass das Gewicht der Dimere in $D'$ in jeder Schleifen  Eins ist, da das für $D_0$
der Fall ist. Damit wäre sichergestellt, dass $\omega(D)=$ für alle $D$.

Für die kleinstmögliche Schleife mit 2 Plätzen ist dies immer erfüllt, wegen $I_3=I_2$ und $I_4=1$
\begin{align*}
    b_{I_1,I_2}b_{I_3,I_4} &=    b_{I_1,I_2}b_{I_2,I_1}= -b_{I_1,I_2}^2 =-1\;.
\end{align*}
Die nächstgrößere Schleife  besteht aus 4 Plätzen und liefert
\begin{align*}
   \omega_Q &=  b_{(i_x,i_y),(i_x,i_y+1)} b_{(i_x,i_y+1),(i_x+1,i_y+1)}b_{(i_x+1,i_y+1),(i_x+1,i_y)}b_{(i_x+1,i_y),(i_x,i_y)}
\end{align*}
Wir hatten uns bereits darauf festgelegt, dass
\begin{align*}
    b_{(i_x,i_y+1),(i_x+1,i_y+1)} = 1\; .
\end{align*}
Deshalb ist der 2. Faktor 1 und der vierte -1. Somit gilt für das elementare Quadrat
\begin{align*}
    \omega_Q &= - b_{(i_x,i_y),(i_x,i_y+1)} b_{(i_x+1,i_y+1),(i_x+1,i_y)}\;.
\end{align*}
Offensichtlich können wir das Hüpfen in y- Richtung nicht auch auf 1 setzen. Stattdessen benötigen wir alternierende Vorzeichen je nach x-Koordinate
\begin{align}\label{eq:b_y}
    b_{(i_x,i_y),(i_x,i_y+1)} &= (-)^i_x\;,
\end{align}
denn dann ist  $\omega_Q=-1$. Wir können nun jede beliebige Schleife dadurch erhalten,dass ausgehend vom elementaren Quadrat sukzessive elementare Quadrate anbauen.
Angenommen wir haben bereits eine bestimmte Schleife $S$ mit dem Gewicht $\omega_S$ berechnet. Nun hängen wir an einer passenden Stelle ein elementares Quadrat an.
Das Quadrat selbst liefert einen zusätzlichen Faktor (-1). Jede Seite, die das elementare Quadrat mit der Schleife S gemeinsam hat, löschen wir aus. Dazu berücksichtigen wir, 
dass die Seiten in S und dem Quadrat in entgegengesetzter Richtung durchlaufen werden und somit jede Seite einen Faktor (-1) liefert. Insgesamt erhalten wir also zum Gewicht von 
S bei $\nu$ gemeinsamen Seiten ein Vorzeichen $(-1)^{(\nu+1)}$. Das sieht zunächst nicht so gut aus. Wir müssen aber berücksichtigen, dass nicht alle Schleifenformen tatsächlich 
durch $D$ und $D'$ erzeugt werden. Wenn im Inneren einer Schleife Gitterpunkte vorkommen, die nicht zur Schleife gehören, dann müssen diese Teil einer (mehrerer) Schleife(n)
sein. Das heißt, die Zahl der inneren Punkte muss gerade sein. Wenn wir nun ein elementares Quadrat hinzufügen ändert sich die Zahl der inneren je nach Zahl der gemeinsamen Kanten

\begin{tabular}{||l|l|l||}
  \hline
  % after \\: \hline or \cline{col1-col2} \cline{col3-col4} ...
  N(gem. Kanten) & $\Delta N$ (innere Punkte) & Änderung des Vorzeichens von $\omega$\\
  \hline
  1 & 0, -2  & +1\\
  2 & 1, -1 & -1\\
  3 & 0, +2  & +1\\
  \hline
\end{tabular}

Ein Wechsel von einer geraden Anzahl innerer Punkte auf eine ungerade oder umgekehrt bringt einen Vorzeichenwechsel in $\omega$ mit sich.
Bleibt die Anzahl der inneren Punkte (un-)gerade ändert sich das Vorzeichen nicht. Daraus folgt,
dass mit gerader Zahl innerer Punkte auch das Vorzeichen der Schleife das vom Ausgangangsquadrat, also -1, ist. Damit haben wir unser Ziel erreicht und 
$I$ ist identisch mit der Zahl der Dimer-Konfigurationen. Wir gehen nun noch einen Schritt weiter und multiplizieren die b-s mit den Variablen $h$ ($v$) je nachdem, ob es
um einen horizontalen oder vertikalen Dimer handelt. Dann ist offensichtlich
\begin{equation}\label{eq:gv5}
    I_2 = \sum_{\cal D} h^{n_h(D)} v^{n_v(D)}\;,
\end{equation}
wobei $n_h(D)$  ($n_v(D)$) die Zahl der horizontalen (vertikalen) Dimere in $D$ ist.
Für die Summe muss gelten $n_h(D) + n_v(D) =N_D$. Wir können $I_2$ nach den möglichen Zahlen $n$ der horizontalen Dimere
umgruppieren
\begin{align*}
    I_2 &= \sum_{\cal D} h^{n_h(D)} v^{N_D - n_h(D)}\; \sum_{n=0}^{N_d} \delta{n,n_h(D)}\\
    &=  \sum_{n=0}^{N_d}  h^n v^{N-n}\;\underbrace{\sum_{\cal D}\;\delta{n,n_h(D)}}_{=:N(n)}\\
    &=  \sum_{n=0}^{N_d}  N(n)\;h^n v^{N-n}\;.
\end{align*}












\section{Auswertung}


Im folgenden vertauschen wir oBdA  die x- und y-Richtung.
Die Matrixelemente des 2d Gitters lauten
\begin{align*}
    M_{i_x,i_y;j_x,j_y} &= \overbrace{\delta_{i_x,j_x} h \cdot M^{(y)}_{i_y,j_y}}^\text{Hüpfen in y-Richtung} +
    \overbrace{v\cdot M^{(x)}_{i_x,j_x} D_{i_y,j_y}}^\text{Hüpfen in x-Richtung}
    \intertext{mit der Diagonalmatrix}
    D_{ij} &:= \delta_{ij} \;(-)^{i}\\
\end{align*}

Hierbei hat  die Matrix $M^{\alpha}$ die Dimension $L_\alpha\times L_\alpha$ und die Form der antisymmetrischen tight-binding MAtrix
\begin{align*}
    M^{\alpha}
    &=
    \begin{pmatrix}
    0&1&0&0&0\\
    -1&0&1&0&0\\
    0&-1&0&1&0\\
    0&0&-1&0&1\\
       0& 0&0&-1&0\\
       &&&&&\ddots
    \end{pmatrix}
\end{align*}


Die Matrixelemente sind die Elemente eins Tensorproduktes
\begin{align*}
    M_{i_x,i_y;j_x,j_y} &= \langle i_x,i_y | M |j_x,j_y\rangle\\
    M   &= v\cdot \EE\otimes M^{(y)} +  h\cdot M^{(x)}\otimes D\\
    M_{i_x,i_y;j_x,j_y} &= v\cdot\langle i_x | \EE |j_x\rangle \langle i_y | M^{y} |j_y\rangle +
    h\cdot \langle i_x | M^{x} |j_x\rangle \langle i_y | D |j_y\rangle
\end{align*}

Es gilt
\begin{align*}
    (D M) _{ij} &= (-)^i M_{ij} = \begin{pmatrix}
    0&-1&0&0&0\\
    -1&0&1&0&0\\
    0&1&0&-1&0\\
    0&0&-1&0&1\\
       0& 0&0&1&0\\
       &&&&&\ddots
    \end{pmatrix}\\
    (M D) _{ij} &=  M_{ij}(-)^j = \begin{pmatrix}
    0&1&0&0&0\\
    1&0&1&0&0\\
    0&-1&0&-1&0\\
    0&0&-1&0&1\\
       0& 0&0&1&0\\
       &&&&&\ddots
           \end{pmatrix}\\
           \qquad\Rightarrow\\
       D M + M D &=0
\end{align*}
Die Matrizen $M$ und $D$ antikommutieren und deshalb gilt zusammen mit $D^2=\EE$
\begin{align*}
    M^2 &= \bigg(v\cdot\EE\otimes M^{(y)} + h\cdot  M^{(x)}\otimes D\bigg)^2\\
    &=v^2\cdot \EE\otimes \big(M^{(y)}\big)^2 + h\cdot \big(M^{(x)}\big)^2\otimes \EE + v\cdot h \cdot M^{(x)}\otimes \underbrace{\big(D M^{y} +M^{y} D\big)}_{=0}\\
    &=v^2\cdot \EE\otimes \big(M^{(y)}\big)^2 +  h^2\cdot \big(M^{(x)}\big)^2\otimes \EE\;.
\end{align*}
Die Eigenvektoren hiervon sind die direkten Produkte der Eigenvektoren der beiden Matrizen $M^{\alpha}$ in den beiden Unterräumen.
Die Eigenwerte ist die Summe der individuellen Eigenwerte multipliziert mit $v^2$ bzw. $h^2$.



Die Eigenvektoren von $M^{\alpha}$ haben die Komponenten  $x_l =  i^l\;\sin{k l}$ liefert für die $l-te$ Gleichung
\begin{align*}
    (M \vec x)_l &= i^{l+1}\sin(k(l+1)) - i^{l-1}\sin(k(l-1))\\
    &= i^{l} \bigg(i \sin(k(l+1)) + i^{1}\sin(k(l-1))\bigg)\\
    &= i i^{l} \bigg(\sin(kl)\cos(k) + \cos(kl)\sin(k) + \sin(kl)\cos(k)-\cos(kl)\sin(k)\bigg)\\
        &= 2 i \cos(k) \;i^{l} \sin(kl)\;.
\end{align*}
Der zugehörige Eigenwert ist demnach $2i\cos(k)$.
Die erste Randgleichung
sind automatisch erfüllt, da für sie dieselben Gleichungen verwendet werden können , wenn $x_0=0$ und $x_{L_\alpha+1}=0$ gefordert wird.
Die erste Bedingung ist immer erfüllt und die letzte fordert $k_\alpha=\pi*n/(L_\alpha+1)$
Die Eigenwerte zum Quadrat sind also
\begin{align*}
    \lambda^2_k := - 4\cos^2(k)
\end{align*}
Damit sind die Eigenwerte von $M^2$ von der Form
\begin{align*}
    \lambda_{n_x,n_y} &= 4 h^2 \cos^2\big(\frac{\pi n_x}{L_x+1}\big) + 
    4 v^2 \cos^2\big(\frac{\pi n_y}{L_y+1}\big)
\end{align*}
Schließlich wissen wir, dass 
\begin{align*}
    I_2 &= {\rm Pf}\big( M\big)\intertext{und somit}
    I_2^4 &= {\rm Pf}^4\big( M\big) = \det\big( M^2\big)\\
    &= \prod_{n_x=1}^{L_x} \prod_{n_y=1}^{L_y} 
    \bigg(    4 h^2 \cos^2\big(\frac{\pi n_x}{L_x+1}\big) +    4 v^2 \cos^2\big(\frac{\pi n_y}{L_y+1}\big)    \bigg)\\
        &= 2^{4 N_D}\;\prod_{n_x=1}^{L_x} \prod_{n_y=1}^{L_y}
    \bigg(    h^2 \cos^2\big(\frac{\pi n_x}{L_x+1}\big) +    v^2 \cos^2\big(\frac{\pi n_y}{L_y+1}\big)    \bigg)\\
    I_2^4 &= {\rm Pf}^4\big( M\big) = \det\big( M^2\big)\\
    &= \prod_{n_x=1}^{L_x} \prod_{n_y=1}^{L_y}
    \bigg(    4 h^2 \cos^2\big(\frac{\pi n_x}{L_x+1}\big) +    4 v^2 \cos^2\big(\frac{\pi n_y}{L_y+1}\big)    \bigg)\\
   I_2        &= 2^{N_D} \left[\;\prod_{n_x=1}^{L_x} \prod_{n_y=1}^{L_y}
    \bigg(    h^2 \cos^2\big(\frac{\pi n_x}{L_x+1}\big) +    v^2 \cos^2\big(\frac{\pi n_y}{L_y+1}\big)    \bigg)\right]^{1/4}\\
\end{align*}

