
\chapter{Appendix}




\section{Sommerfeld expansion}\label{Sommerfeld}
We are interested in integrals of the form
%
\begin{align*}
I&=\int_{0}^{\infty}  f(\varepsilon) n_{F}(\varepsilon|\mu,T)\; d\varepsilon
\end{align*}
%
for low temperatures. The function $f(\varepsilon)$ is  assumed to be independent of $T$
and regular at $\varepsilon=\mu$.
First we split the integral as follows
%
\begin{align*}
I&=\int_{0}^{\mu }  f(\varepsilon) \frac{1}{e^{\beta(\varepsilon-\mu)}+1}\; d\varepsilon
+
\int_{\mu}^{\infty }  f(\varepsilon) \frac{1}{e^{\beta(\varepsilon-\mu)}+1}\;  d\varepsilon\;,
\end{align*}
%
and modify the first term using $1/(e^{x}+1)=1-1/(e^{-x}+1)$
\begin{align*}
I&=\underbrace{
\int_{0}^{\mu }  f(\varepsilon)\;d\varepsilon
}_{\color{blue} = I_{1}}
-\underbrace{
\int_{0}^{\mu }  f(\varepsilon) \frac{1}{e^{-\beta(\varepsilon-\mu)}+1}\; d\varepsilon
}_{\color{blue} = I_{2}}
+
\underbrace{
\int_{\mu}^{\infty }  f(\varepsilon) \frac{1}{e^{\beta(\varepsilon-\mu)}+1}\;  d\varepsilon
}_{\color{blue} = I_{3}}
\end{align*}
In $I_{2}$ we use the transformation 
%
\begin{align*}
x &= -\beta(\varepsilon-\mu)\;,\\
\varepsilon &= \mu- k_{B}T \;x\;,\\
d\varepsilon &= -k_{B}T d x\;,
\end{align*}
%
and find
%
\begin{align*}
I_{2} &= -k_{B}T\int_{0}^{\beta\mu} \frac{f(\mu-k_{B}T x)}{e^{x}+1}\;.
\end{align*}
%
The kernel $1/(e^{x}-1)$ decreases rapidly with $x$; and for small values of  $T$ the upper integration limit $\beta \mu$ is much greater than 1, so we can as well extend the integral all the way up 
to infinity
\begin{align*}
I_{2} &= -k_{B}T\int_{0}^{\infty} \frac{f(\mu-k_{B}T x)}{e^{x}+1}\;.
\end{align*}
%
For $I_{3}$ we introduce the substitution 
\begin{align*}
x &= \beta(\varepsilon-\mu)\;,\\
\varepsilon &= \mu+ k_{B}T \;x\;,\\
d\varepsilon &= k_{B}T d x\;,
\end{align*}
%
and obtain
%
\begin{align*}
I_{3}&= k_{B}T\int_{0}^{\infty} \frac{f(\mu+k_{B}T x)}{e^{x}+1}\;.
\end{align*}
%
Since $1/(e^{x}+1)$ decreases rapidly, $x$ is essentially restricted to $x\lessapprox 1$. 
Hence $k_{B}T x\lessapprox k_{B}T$. So if $k_{B}T/\mu$ is small, then we can use a Taylor expansion
%
\begin{align*}
f(\mu+ s k_{B}T x) &= \sum_{n=0}^{\infty} s^{n}\frac{(k_{B}T x)^{n}}{n!} f^{(n)}(\mu)\;.
\end{align*}
%
We need 
%
\begin{align*}
I_{3}-I_{2} &=k_{B}T\int_{0}^{\infty}dx 
\bigg( f(\mu+k_{B}Tx) - f(\mu-k_{B}Tx) \bigg)\frac{1}{e^{x}+1}\\
 &=2  \sum_{n=1}^{\text{odd}}
\frac{\big(k_{B}T\big)^{n+1}}{n!} f^{(n)}(\mu)\;
\int_{0}^{\infty}dx
 \frac{x^{n}}{e^{x}+1}\\
  &=2  \sum_{n=1}^{\text{odd}}
\frac{\big(k_{B}T\big)^{n+1}}{n!}\;f^{(n)}(\mu)\;
\big(1-\frac{1}{2^{n}}\big)\Gamma(n+1)\zeta(n+1)\;.
% &=2  \sum_{n=1}^{\text{odd}}\big(1-\frac{1}{2^{n}}\big)\;\zeta(n+1)
%\big(k_{B}T\big)^{n+1}\;f^{(n)}(\mu)\;
\end{align*}
%
The final result reads
%
\tboxit{Sommerfeld expansion}{
\begin{align}\label{eq:}
\int_{0}^{\infty}  f(\varepsilon) n_{F}(\varepsilon|\mu,T) d\varepsilon
 &=\int_{0}^{\mu} f(\varepsilon)d\varepsilon + 2  \sum_{n=1}^{\text{odd}}\big(1-\frac{1}{2^{n}}\big)\;\zeta(n+1)
\big(k_{B}T\big)^{n+1}\;f^{(n)}(\mu)\;
\end{align}}
%
So the leading order terms in the Sommerfeld expansion are
%
\begin{align*}
I &= \int_{0}^{\mu} f(\varepsilon) d\varepsilon +  \frac{\pi^{2}}{6}\;\big(k_{B}T\big)^{2} \;f'(\mu)+ {\cal O}(\big(\frac{k_{B}T}{\mu}\big)^{4})
\end{align*}
%
%=========================================================
\section{Euler Mac-Laurin formula}\label{app:Euler:MacLaurin}
We are interested in sums of the form
$\sum_{n=0}^{\infty} f(n+\tfrac{1}{2})$.
We begin with the integral over $f(x)$, which shall be differentiable at least twice and the integral over $(0,\infty)$ shall exist. Then
%
%
\begin{align*}
\int_{0}^{\infty} f(x) dx
&=\sum_{n=0}^{\infty}\int_{n}^{n+1}  f(x)dx
=\sum_{n=0}^{\infty}\int_{-\tfrac{1}{2}}^{\tfrac{1}{2}}
  f(n+\tfrac{1}{2} + \xi)d\xi\\
&=
  \sum_{n=0}^{\infty}\int_{-\tfrac{1}{2}}^{\tfrac{1}{2}}
  \bigg(f(n+\tfrac{1}{2}) + f'(n+\tfrac{1}{2}) \xi
  +f''(n+\tfrac{1}{2}) \frac{\xi^{2}}{2} +\ldots \bigg)d\xi\\
  &=
  \sum_{n=0}^{\infty}
  \bigg(f(n+\tfrac{1}{2})\int_{-\tfrac{1}{2}}^{\tfrac{1}{2}} d\xi + f'(n+\tfrac{1}{2}) 
  \int_{-\tfrac{1}{2}}^{\tfrac{1}{2}} 
  \xi d\xi
  +\frac{f''(n+\tfrac{1}{2})}{2} \int_{-\tfrac{1}{2}}^{\tfrac{1}{2}}\;\xi^{2} d\xi+\ldots \bigg)d\xi\\
&=
  \sum_{n=0}^{\infty}
  \bigg(f(n+\tfrac{1}{2}) 
  +f''(n+\tfrac{1}{2})\frac{2}{2\cdot 3 \cdot 2^{3}}+\ldots \bigg)d\xi\;.
\end{align*}
The leading order terms are
%
\begin{align*}
\sum_{n=0}^{\infty} f(n+\tfrac{1}{2}) &= \int_{0}^{\infty}
f(x) dx - \frac{1}{24} \sum_{n=0}^{\infty} f''(n+\tfrac{1}{2})+\ldots
\end{align*}
%
We can use this formula again to express the sum on the rhs also by an integral, resulting in 
\begin{align*}
\sum_{n=0}^{\infty} f(n+\tfrac{1}{2}) &= 
\int_{0}^{\infty}f(x) dx 
- \frac{1}{24} \bigg(\int_{0}^{\infty}f''(x) dx  -\frac{1}{24} \sum_{n=0}^{\infty} f^{(iv)}(n+\tfrac{1}{2})\bigg)+\ldots\;,
\end{align*}
%
and we obtain 

%
\tboxitp{Euler-MacLaurin formula}{leading order}{
\begin{align}\label{eq:Euler:MacLaurin}
\sum_{n=0}^{\infty} f(n+\tfrac{1}{2}) &= 
\int_{0}^{\infty}f(x) dx  +\frac{1}{24} f'(0)-\frac{1}{24} f'(\infty)  + \ldots 
\end{align}}
%
These are actually the first terms of the {\em Euler MacLaurin formula} formula.


%=========================================================

\section{One-particle density of states } \label{app:dos}

The one-particle density of states in D spatial dimensions is
%
\begin{align*}
\rho^{(D)}(\varepsilon) &= \sum_{\vv k}\delta(\varepsilon-\frac{\hbar^{2} \vv k^{2}}{2m})\\
 &= \frac{V^{(D)}}{(2\pi)^{D}}\int d^{D}\vv k\; \delta\bigg(\varepsilon-\bigg(\frac{\hbar \vv k}{\sqrt{2m}}\bigg)^{2}\bigg)\\
&=\frac{V^{(D)}}{(2\pi)^{D}}\bigg( \frac{\sqrt{2m}}{\hbar}\bigg)^{D}\int d^{D}\vv x \delta(\varepsilon-\vv x^{2})\\
&=\frac{ V^{(D)} (2m)^{D/2}}{(2\pi)^{D} \hbar^{D}}\;\Omega_{D} \;\int_{0}^{\infty} d x x^{D-1}\delta(\varepsilon-x^{2})\\
&=\frac{V^{(D)} (2m)^{D/2}}{(2\pi)^{D} \hbar^{D}}\;\Omega_{D}
\;\theta(\varepsilon\ge 0) \;\int_{0}^{\infty} d x x^{D-1}\frac{\delta(x-\sqrt{\varepsilon})}{2 x}\\
&=\frac{V^{(D)} (2m)^{D/2}}{(2\pi)^{D} \hbar^{D}}\;\frac{\Omega_{D}}{2} \;
\varepsilon^{\frac{D-2}{2}}\;\theta(\varepsilon\ge 0)\\
\end{align*}
%
In appendix C of the statistics I script we have derived the surface $\Omega_{D}$ of a
a $D$-dimensional hypersphere of unit radius
%
\begin{align*}
\Omega_{D} &= \frac{2\pi^{D/2}}{\Gamma(D/2)}\;.
\end{align*}
%
The D-dimensional dos of the free electron gas is therefore
%
\tboxitp{Density of states}{D-dimensional free electron gas}{
\begin{subequations}
\begin{align}\label{eq:}
\rho^{(D)}(\varepsilon) &=
\frac{V^{(D)} m^{\frac{D}{2}} }{\hbar^{D} (2\pi)^{\frac{D}{2}}\Gamma\big(\frac{D}{2}\big)}\;
\;\varepsilon^{\frac{D-2}{2}}\;\theta(\varepsilon\ge 0)\;.\\
\rho^{(1)}(\varepsilon) &=
\frac{L \sqrt{m} }{\hbar \sqrt{2} \pi}\;
\;\frac{1}{\sqrt{\varepsilon}}\;\theta(\varepsilon\ge 0)\;.\\
\rho^{(2)}(\varepsilon) &=
\frac{V^{(2)} m }{\hbar^{2} \;2\;\pi}\;\theta(\varepsilon\ge 0)\;,\\
\rho^{(3)}(\varepsilon) 
%&=\frac{V m^{\frac{3}{2}} }{\hbar^{3} (2\pi)^{\frac{3}{2}} \sqrt{\pi}/2}\;
%\;\sqrt{\varepsilon}\;.\\
&=\frac{V m^{\frac{3}{2}} }{\hbar^{3} \sqrt{2} \pi^{2}}\;
\;\sqrt{\varepsilon}\;\theta(\varepsilon\ge 0)\;.
\end{align}
\end{subequations}}
%

\section{Eigenvalues of the free electron gas in a homogeneous magnetic field}
Here we determine the one-particle eigenvalues of
 the free electron gas in a homogeneous magnetic field. Since the electron do not interact
 with each other, it suffices to consider the hamiltonian of  single electron, which
reads
%
\begin{align*}
H &= \frac{\big( \hat{\vv p} +e \hat{\vv A} \big)^{2}}{2m} + \mu_{B} g_{e} \vv B \hat{\vv S}\;,
\end{align*}
%
where $\vv S$ is the vector operator of the electronic spin. The direction of the magnetic field
field defines the $z$-direction. Since there is no spin-orbit coupling, the eigenvector is a tensor product of the  orbital and spin degrees of freedom. For the letter the vector is the eigenvector
of the operator $S_{z}$, i.e.
%
\begin{align*}
\ket{\psi} &= \ket{\Phi}\otimes \ket{\sigma}
\end{align*}
%
The eigenvalue problem turns into
%
\begin{align*}
H\ket{\psi} &= \ket{\sigma} \otimes\bigg( \frac{\big( \hat{\vv p} +e \hat{\vv A} \big)^{2}}{2m} 
+ \underbrace{
\frac{\mu_{B}\hbar g_{e}}{2}
}_{\color{blue} = b} \sigma \bigg)\ket{\Phi} = \varepsilon \ket{\sigma}\otimes\ket{\Phi}
\end{align*}
%
The orbital part of the eigenvalue problem reads
\begin{align*}
 \frac{\big( \hat{\vv p} +e \hat{\vv A} \big)^{2}}{2m} \ket{\Phi} =\varepsilon' \ket{\Phi}
\end{align*}
and the the eigenvalues of the entire hamitonian are
%
\begin{align*}
\varepsilon &= \varepsilon' + b\sigma\;.
\end{align*}
%
The vector potential for the homogeneous magnetic field in $z$-direction can be chosen (Landau gauge) as
%
\begin{align*}
\vv A &= B (0,x,0)\;.
\end{align*}
%
We readily see that it gives the correct $\vv B$ field
%
\begin{align*}
\vv B &=\nabla \times \vv A \\
&= 
\begin{pmatrix}
	\partial_{y} A_{z}  - 	\partial_{z} A_{y}\\
		\partial_{z} A_{x}  - 	\partial_{x} A_{z}\\
			\partial_{x} A_{y}  - 	\partial_{y} A_{x}
\end{pmatrix}\\
&= B 
\begin{pmatrix}
	0\\
	0\\
	1
\end{pmatrix}\;.
\end{align*}
%
Inserting $\vv A$ into the orbital eigenvalue problem gives
%
\begin{align*}
 \frac{\hat p_{x}^{2}+(\hat p_{y}+e B \hat x)^{2}+\hat p_{z}^{2}}{2m} \ket{\Phi} =\varepsilon' \ket{\Phi}
\end{align*}
%
The $y$ and $z$ coordinate only enters via the momenta, therefore, the eigenvector is a tensor
product of the form
%
\begin{align*}
\ket{\Phi} &= \ket{\Phi_{x}}\otimes\ket{p_{y}}\otimes\ket{p_{z}}\;,
\end{align*}
%
where $\ket{k_{x}}$ ($\ket{k_{x}}$) is an eigenvector of the momentum operator $p_{x}$
($p_{y}$) with the corresponding eigenvalue. Then
\begin{align*}
 \frac{\hat p_{x}^{2}+(\hat p_{y}+e B \hat x)^{2}+\hat p_{z}^{2}}{2m}   \ket{\Phi}&=
\ket{p_{y}}\otimes\ket{p_{z}}\otimes \frac{\hat p_{x}^{2}+(p_{y}+e B \hat x)^{2}+p_{z}^{2}}{2m} \ket{\Phi_{x}} \\
&=\ket{p_{y}}\otimes\ket{p_{z}}\otimes \bigg(\varepsilon' \ket{\Phi_{x}}\bigg)
\end{align*}
The remaining 1D problem reads
%
\begin{align*}
 \frac{\hat p_{x}^{2}+(p_{y}+e B \hat x)^{2}+p_{z}^{2}}{2m} \ket{\Phi_{x}} 
&=\varepsilon' \ket{\Phi_{x}}
\end{align*}
%
The electron gas is confined to a box of size $L_{x}\times L_{y}\times L_{y}$. The momentum eigenvalues are therefore quantized as
%
\begin{align}
p_{y} &= \hbar \frac{2\pi}{L_{y}} n_{y}\;,\qquad \text{with } n_{y}\in \mathbf N \\
p_{z} &= \hbar \frac{2\pi}{L_{z}} n_{z}\;,\qquad \text{with } n_{z} \in \mathbf N\\
\end{align}
%
The 1D hamiltonian can be rewritten in the form
\begin{align*}
\bigg( \frac{1}{2m}\hat p_{x}^{2}+ \frac{e B}{2m} (\hat x + \underbrace{
\frac{p_{y}}{eB}
}_{\color{blue} := -x_{0}} )^{2}+
 \frac{p_{z}^{2}}{2m} \bigg)\ket{\Phi_{x}} 
&=\varepsilon' \ket{\Phi_{x}}
\end{align*}
This is the hamiltonian of a harmonic oscillator shifted by $x_{0}$. 
The prefactor of the $x^{2}$ term corresponds to $\omega_{c}/2$, hence $\omega_{c}=eB/m$. 
The  eigenvalues of the harmonic oscillator yield
%
\begin{align*}
\varepsilon' &=  \frac{p_{z}^{2}}{2m}  + \hbar \omega_{c} \big( n + \tfrac12 \big)\;.
\end{align*}
%
These eigenvalues are independent of $p_{y}$ and, therefore, degenerate. $p_{y}$ defines the 
center $x_{0}$ of the harmonic oscillator, which has to be within $(0,L_{x})$, resulting in the condition
%
\begin{align*}
0\le \frac{p_{y}}{e B} \le L_{x} \\
0\le \frac{\hbar 2\pi }{L_{y} e B}n_{y} \le L_{x} \\
0\le n_{y} \le \frac{L_{x} L_{y} e B}{ 2\pi\hbar}\;.
\end{align*}
%
The number of allowed $p_{y}$ values defines the degeneracy
%
\begin{align*}
N_\text{deg} &= \bigg\lfloor\frac{L_{x} L_{y} e B}{ 2\pi\hbar }\bigg\rfloor\; +1.
\end{align*}
%
This number will be much greater than 1 and we can, therefore, ignore the fact that it has to be an integer and we simply use
\begin{align}
N_\text{deg} &= \frac{L_{x} L_{y} e B}{ 2\pi\hbar }\;.
\end{align}

\section{Proof of the Mermin-Wagner Theorem\label{app:mermin:wagner}}
The proof is based on the
\tboxit{Bogoliubov inequality}{
%
\begin{align}\label{eq:}
\frac{\beta}{2} \avg{[A,A^{\dagger}]_{+}}\;
\avg{\big[[C,H]_{-},C^{\dagger}\big]_{-}} &\ge \abs{[C,A]_{-}}^{2}
\end{align}}
%
the proof of which is given in \app{app:Bogoliubov}.

Now we start with the Mermin-Wagner theorem, which is valid for the 
following  hamiltonian 
 %
\begin{align}\label{eq:}
H &= -\sum_{jj'}J_{jj'} \vv S_{j} \vv S_{j'} + b    S^{z}_{\vv Q}\;.
\end{align}
%
%
The division by $\hbar$ makes the order parameter dimensionless and the factor
$1/N$ leads to the magnetization per site in the ferromagnetic case.
%
The  spin operators obey the common commutator relations of angular momenta
\begin{subequations}\label{eq:}
\begin{align}
\big[ S^{z}_{j} S^{\pm}_{j'}\big] &= \pm \delta_{jj'} \hbar S_{j}^{\pm}\\
\big[ S^{+}_{j} S^{-}_{j'}\big] &= \delta_{jj'} 2 \hbar S_{j}^{z}\;.
\end{align}
\end{subequations}
%
It is expedient to introduce Fourier transformed operators

\begin{subequations}\label{eq:FT:magnetism}
\begin{align}
S_{k}^{\alpha} &= \sum_{j} S_{j}^{\alpha} e^{i \vv k \vv R_{j}}\;,\qquad \alpha\in\{x,y,z\}\\
S_{j}^{\alpha}&=\frac{1}{N} \sum_{k}^{1.Bz} S_{k}^{\alpha} e^{-i \vv k \vv R_{j}}
\end{align}
\end{subequations}
%
From that we obtain
%
\begin{align}
S_{\vv k}^{\pm} &:= \sum_{j} e^{i\vv k \vv R_{j}} S_{j}^{\pm}
=\sum_{j} e^{i\vv k \vv R_{j}}\big(S_{j}^{x}+i S_{j}^{y}\big)
= S_{\vv k}^{x} \pm i S_{\vv k}^{y}
\end{align}
%
The commutation relations in these operators are
\begin{subequations}\label{eq:}
\begin{align}
\big[ S^{z}_{k} S^{\pm}_{k'}\big] &= \pm \hbar S_{k+k'}^{\pm}\\
\big[ S^{+}_{k} S^{-}_{k'}\big] &= 2 \hbar S_{k+k'}^{z}\;.
\end{align}
\end{subequations}
%

We assume that the exchange coupling decays rapidly enough to ensure
%
\begin{align}\label{eq:X}
X&:=\frac{S(S+1)\hbar^{2}}{N} \sum_{jj'} \abs{J_{jj'}} \abs{\vv R_{j}-\vv R_{j'}}^{2} < \infty\;,&&\forall N\;.
\end{align}
%
For the proof of the Mermin-Wagner theorem the following operators are used
%
\begin{subequations}\label{eq:MW:operators}
\begin{align}\label{eq:}
A = A_{\vv k}&=S^{-}_{\vv Q-\vv k}\;,&A^{\dagger}_{\vv k} &=S^{+}_{\vv k-\vv Q}\\
C = C_{\vv k} &=S^{+}_{\vv k}\;,&C^{\dagger}_{\vv k} &=S^{-}_{-\vv k}\;.
\end{align}
\end{subequations}
%
The first term in  the Bogoliubov inequality that we compute is
%
\begin{align}\label{eq:MW:1}
\avg{\big[ C_{\vv k},A_{\vv k} \big]_{-}}  &= \avg{\big[ S^{+}_{\vv k},S^{-}_{\vv Q-\vv k} \big]}
= 2 \hbar N\;{\cal S}^{z}_{\vv Q}(T,b) \;.
\end{align}
%
The second term  that we consider is
%
\begin{align}\label{eq:}
\sum_{\vv k} \avg{\big[ A_{\vv k},A_{\vv k}^{\dagger} \big]_{+} } 
&=\sum_{\vv k}\avg{  S^{-}_{\vv Q-\vv k} S^{+}_{\vv k - \vv Q}
+S^{+}_{\vv k - \vv Q} S^{-}_{\vv Q-\vv k}}\\
&=\sum_{jj'}\sum_{\vv k} e^{i(\vv Q - \vv k)\vv R_{j}}e^{i(\vv k - \vv Q )\vv R_{j'}}\avg{  S^{-}_{j} S^{+}_{j'}
+S^{+}_{j'} S^{-}_{j}}\\
&=\sum_{jj'} e^{i \vv Q (\vv R_{j}- \vv R_{j'})}
\underbrace{
\sum_{\vv k} e^{-i \vv k(\vv R_{j}-\vv R_{j'})} 
}_{\color{blue} = N \delta_{jj'}}
\avg{  S^{-}_{j} S^{+}_{j'}
+S^{+}_{j'} S^{-}_{j}}\\
&=N \;\sum_{j} 
\avg{  S^{-}_{j} S^{+}_{j}
+S^{+}_{j} S^{-}_{j}}\\
%%%
&=2N\sum_{j} \avg{ \big(S^{x}_{y}\big)^{2}+\big(S^{y}_{j}\big)^{2}}\;,
\end{align}
%
This term can be estimated by
%
\begin{align*}
\sum_{\vv k} \langle\big[ A_{\vv k},A_{\vv k}^{\dagger} \big]_{+}  \rangle 
&\le 2N\sum_{j}\; \underbrace{
\avg{ \big(\vv S_{j}\big)^{2}}
}_{\color{blue} = \hbar^{2}S(S+1)}
=2N^{2}\hbar^{2}S(S+1)\;,
\end{align*}
%
resulting in
%
\begin{align}\label{eq:MW:2}
 \beta N^{2} \hbar^{2} S(S+1)&\ge
  \frac{\beta}{2}\sum_{\vv k} \langle\big[ A,A^{\dagger} \big]_{+}  \rangle  \;,
\end{align}
%
For the remaining term in the Bogoliubov inequality we find (proof in appendix \app{app:prove:Bogoliubov:3})
%
\begin{align}\label{eq:MW:3}
4 \hbar^{2} N\big( \abs{b {\cal S}_{\vv Q}(T,b)} +  \vv k^{2} X \big)\ge 
 \avg{\bigg[ \big[C_{\vv k},H  \big]_{-},C^{\dagger}_{\vv k} \bigg]_{-}}
\end{align}
%
with the definition of $X$ in \eq{eq:X}.
The Bogoliubov inequality reads
%
\begin{align*}
\frac{\beta}{2} \avg{[A_{\vv k},A_{\vv k}^{\dagger}]_{+}}\;
\avg{\big[[C_{\vv k},H]_{-},C_{\vv k}^{\dagger}\big]_{-}} &\ge \abs{\avg{[C_{\vv k},A_{\vv k}]_{-}}}^{2}
\end{align*}
%
According to \eq{eq:schwarz:3} the second factor is greater zero, therefore
%
\begin{align*}
\frac{\beta}{2} \avg{[A_{\vv k},A_{\vv k}^{\dagger}]_{+}}\;
 &\ge \frac{\abs{\avg{[C_{\vv k},A_{\vv k}]_{-}}}^{2}}{ \avg{\big[[_{\vv k}C,H]_{-},C_{\vv k}^{\dagger}\big]_{-}}}\\
\Rightarrow\qquad \frac{\beta}{2}\sum_{\vv k} \avg{[A_{\vv k},A_{\vv k}^{\dagger}]_{+}}\;
 &\ge\sum_{\vv k} \frac{\abs{\avg{[C_{\vv k},A_{\vv k}]_{-}}}^{2}}{ \avg{\big[[C_{\vv k},H]_{-},C_{\vv k}^{\dagger}\big]_{-}}}\;.
\end{align*}
%
Finally, we use the operators in \eq{eq:MW:operators} in the Bogoliubov inequality, sum over
$\vv k$ and use \eq{eq:MW:1},\eq{eq:MW:2} and \eq{eq:MW:3}, resulting in
%
\begin{align*}
 \beta N^{2} \hbar^{2} S(S+1) &\ge 
   \frac{\beta}{2}\sum_{\vv k} \langle\big[ A_{\vv k},A_{\vv k}^{\dagger} \big]_{+}  \rangle  \\
 & \ge\sum_{\vv k} \frac{\abs{[C_{\vv k},A_{\vv k}]_{-}}^{2}}{ \avg{\big[[C_{\vv k},H]_{-},C_{\vv k}^{\dagger}\big]_{-}}}\\
 %%%%%
     \beta N^{2} \hbar^{2} S(S+1)   &\ge \frac{4\hbar^{2} N^{2} {\cal S}^{2}_{\vv Q}(T,b)}{4  N} \;\sum_{\vv k} \frac{1}{
   \abs{b {\cal S}_{\vv Q}(T,b)} +  \vv k^{2}   X }\\
%%%
 \beta  S(S+1)   &\ge \frac{{\cal S}_{\vv Q}(T,b)}{ N} \;\sum_{\vv k} \frac{1}{
   \abs{b {\cal S}_{\vv Q}(T,b)} +  \vv k^{2}   X }\\
 \beta  S(S+1)       &\ge {\cal S}^{2}_{\vv Q}(T,b)\;I\;,
\end{align*}
%
with
%
\begin{align*}
I&:= \frac{1}{N}\sum_{\vv k} \frac{1}{
   \abs{b {\cal S}_{\vv Q}(T,b)} +  \vv k^{2}  X }\\
   &=\frac{V}{N (2\pi)^{D}} \int_{V_{1bc}} d^{D}k\;
   \frac{1}{
   \abs{b {\cal S}_{\vv Q}(T,b)} +  \vv k^{2} X }
\end{align*}
%
The volume $V_{1bc}$ of integration is the first Brillouin zone of a simple square lattice, i.e.
a $D$-dimensional cube with edges ranging from $-\pi$
to $\pi$. Since the integrand is positive, the integral  becomes smaller, if we replace the volume
by that of a sphere, $V_{sp}$, that lies entirely within $V_{1bc}$, i.e,.
%
\begin{align*}
I&\ge I_{sp}\\
I_{sp}&=\frac{V}{N (2\pi)^{D}} \int_{V_{sp}} d^{D}k\;
   \frac{1}{
   \abs{b {\cal S}_{\vv Q}(T,b)} +  \vv k^{2} X}\\
   &=
    \frac{V \Omega_{D}}{N (2\pi)^{D}}\;\int_{0}^{k_{0}} dk
   \frac{k^{D-1}}{
   \abs{b {\cal S}_{\vv Q}(T,b)} +  \vv k^{2} X}\\
   &=
    \frac{const}{\abs{b {\cal S}_{\vv Q}(T,b)}} \;\int_{0}^{k_{0}} \frac{dk}{k}
   \frac{k^{D}}{
1 +  \underbrace{
\vv k^{2}\frac{X}{   \abs{b {\cal S}_{\vv Q}(T,b)}}
}_{\color{blue} := x^{2}}}\\
   &=
    \frac{const}{\abs{b {\cal S}_{\vv Q}(T,b)}} \;
\bigg(    \frac{   \abs{b {\cal S}_{\vv Q}(T,b)}}{X}\bigg)^{D/2}
    \;\int_{0}^{k_{0}\sqrt{\frac{X }{   \abs{b {\cal S}_{\vv Q}(T,b)}}} }\frac{dx}{x}
   \frac{x^{D}}{
1 +  x^{2}}\\
   &=
const \;
  \abs{b {\cal S}_{\vv Q}(T,b)}^{\frac{D-2}{2}}
    \;\int_{0}^{k_{0}\sqrt{\frac{X }{   \abs{b {\cal S}_{\vv Q}(T,b)}}} }
\frac{dx}{x}
   \frac{x^{D}}{
1 +  x^{2}}\;.
\end{align*}
%
So finally we have
%
\begin{align*}
S(S+1) 
   &\ge const \; k_{B}T
\frac{{\cal S}^{2}_{\vv Q}(T,b)}{\hbar^{2}}
  \abs{b {\cal S}_{\vv Q}(T,b)}^{\frac{D-2}{2}}
    \;\int_{0}^    {k_{0}\sqrt{\frac{X }{   \abs{b {\cal S}_{\vv Q}(T,b)}}} }
\frac{dx}{x}
   \frac{x^{D}}{1 +  x^{2}}\;,
\end{align*}
%
or alternative
\begin{align}\label{eq:ineq}
1  &\ge const \;\cdot  T\cdot 
\bigg|{\cal S}_{\vv Q}(T,b)\bigg|^{\frac{D+2}{2}}
  \abs{b}^{\frac{D-2}{2}}
    \;\underbrace{
\int_{0}^{k_{0}\sqrt{\frac{X }{   \abs{b {\cal S}_{\vv Q}(T,b)}}} }\frac{dx}{x}
   \frac{x^{D}}{
1 +  x^{2}}
}_{\color{blue} = K_{D}}\;.
\end{align}
%
We are interested in the spontaneous magnetizaton $b\to 0$, i.e. the upper integration limit goes to $\infty$.

\subsubsection{1D}
%
In the 1D case, the integral for $b\to 0$ becomes
%
\begin{align*}
K_{1}&=\int_{0}^{k_{0}\sqrt{\frac{X }{   \abs{b {\cal S}_{\vv Q}(T,b)}}} }dx 
   \frac{1}{1 +  x^{2}}\underset{b\to 0}{ \longrightarrow }
\int_{0}^{\infty}dx  \frac{1}{1 +  x^{2}}    = \frac{\pi}{2}
\end{align*}
%
Hence 
%
\begin{align*}
1&\ge  const \;T \bigg({\cal S}_{\vv Q}(T,b)\bigg)^{\frac{3}{2}}
  \abs{b}^{-\frac{1}{2}}\\
\frac{  \abs{b}}{T^{2}}&\ge const \;\bigg( {\cal S}_{\vv Q}(T,b)\bigg)^{3}\\
{\cal S}_{\vv Q}(T,b)&\le \bigg(\frac{  \abs{b}}{T^{2}}\bigg)^{1/3} \;.
\end{align*}
%
For finite temperature, the rhs tends to zero for $b\to 0$.
Hence for 1D and finite $T$
%
\begin{align}\label{eq:}
\lim_{b\to 0}{\cal S}_{\vv Q}(T,b)&= 0 \;,
\end{align}
%
i.e. there is no spontaneous magnetization, irrespective of $\vv Q$, for finite $T$.
\red{The inequality does not rule out a finite spontaneous magnetization for $T=0$.}
\subsubsection{2D}
%
In the 2D case, the integral $K_{D}$  is a bit more tricky as it diverges for $b\to 0$
%
\begin{align*}
K_{2}&=\int_{0}^{k_{0}\sqrt{\frac{X }{   \abs{b {\cal S}_{\vv Q}(T,b)}}} }dx 
   \frac{x}{1 +  x^{2}}\\
   &=\frac{1}{2}\int_{0}^{\frac{k_{0}^{2} X}{  \abs{b {\cal S}_{\vv Q}(T,b)}}}  dz
   \frac{1}{1 +  z}\\
   &=\frac{1}{2}\;\ln\big( 1+\frac{k_{0}^{2} X}{  \abs{b {\cal S}_{\vv Q}(T,b)}} \;.
\end{align*}
%
For $b\to 0$ it behaves like
%
\begin{align*}
  K_{2} &=\frac{1}{2}\;\ln\big( \frac{k_{0}^{2} X}{  \abs{b {\cal S}_{\vv Q}(T,b)}} \big)
  = -\frac{1}{2}\;\ln\big( 
  \frac{  \abs{b {\cal S}_{\vv Q}(T,b)}} {k_{0}^{2} X}
  \big)
  = \frac{1}{2}\;
  \abs{\ln\big( 
  \big|\frac{b}{b_{0}}\big|\big)}\;,
\end{align*}
%
where we have defined the positive parameter $b_{0}= k_{0}^{2} X / |S_{\vv Q}(T,b)|$ and we have used that $b\to 0$ result in a negative logarithm.
%
Then  the inequality \eq{eq:ineq} yields 
%
\begin{align*}
1&\ge \frac{const}{2} \;T \bigg({\cal S}_{\vv Q}(T,b)\bigg)^{2}
  \abs{b}^{0} \;\big|  \ln\big( \abs{b/b_{0}}  \big)\big|
  \\
   \bigg({\cal S}_{\vv Q}(T,b)\bigg)^{2}&\le\frac{const}{T
  \big|  \ln\big( \abs{b/b_{0}}  \big)\big| 
   } \;.
\end{align*}
%
The story is the same as in 1D. The rhs goes to zero for $b\to 0$ and finite T.
\red{Nothing can be said about $T=0$.}

\subsubsection{3D}
%
Eventually in the 3D case, the integral $K_{D}$ yields
%
\begin{align*}
K_{3} &= \int_{0}^{k_{0}\sqrt{\frac{X }{   \abs{b {\cal S}_{\vv Q}(T,b)}}} }dx 
   \frac{x^{2}}{1 +  x^{2}}\\
    &= \int_{0}^{k_{0}\sqrt{\frac{X }{   \abs{b {\cal S}_{\vv Q}(T,b)}}} } dx 
\bigg(   1-\frac{1}{1 +  x^{2}}\bigg)\\
&=\sqrt{\frac{c}{ \abs{b {\cal S}_{\vv Q}(T,b)}}}  -\underbrace{
 K_{1D}
}_{\color{blue} = \pi/2}
\end{align*}
%
Inserting in the inequality \eq{eq:ineq} yields
%
\begin{align*}
1&\ge const \;T 
\bigg({\cal S}_{\vv Q}(T,b)\bigg)^{\frac{5}{2}}
  \abs{b}^{\frac{1}{2}}
    \;
    \bigg(\frac{c}{\sqrt{  \abs{b {\cal S}_{\vv Q}(T,b)}}}  - \canceled{\frac{\pi}{2}}  \bigg)\\
&    \ge const \;T 
\bigg({\cal S}_{\vv Q}(T,b)\bigg)^{2}\\
\bigg({\cal S}_{\vv Q}(T,b)\bigg)^{2} &\le  \frac{const}{T}
\end{align*}
%
In this case we do not obtain  a useful inequality. \red{In other words, the Mermin-Wagner theorem 
does not rule out spontaneous magnetization for finite $T$ in 3D.}


%%%%%

\subsection{Bogoliubov inequality }\label{app:Bogoliubov}

Here we will prove the Bogoliubov inequality.
To this end we will exploit the Schwarz inequality
%
\begin{align}\label{eq:schwarz:a}
\abs{\big( A|B \big)}^{2} \le \big( A|A \big)\big( B|B \big)\;,
\end{align}
%
where $(X|Y)$ are scalar products between any  vectors $A$ and $B$.
To be able to use this relation, we have to introduce a suitable scalar product, that 
involves the operators $A$ and $B$
%
\begin{align*}
\big( A|B \big) &:=\sum^{E_{m}\ne E_{n}}_{n,m}
\bra{m} A^{\dagger} \ket{n} \bra{n} B\ket{m}
\frac{\rho_{m}-\rho_{n}}{E_{n}-E_{m}}
\intertext{with}
\rho_{m} &=\frac{e^{-\beta E_{m}}}{\tr{e^{-\beta H}}}\;.
\end{align*}
%.
This product fulfils the four defining properties of a scalar product
\begin{itemize}
	\item $\big( A|B \big)  = \big( B|A \big)^{*} $.

This is obviously the case
%
\begin{align*}
\big( B|A\big)^{*} &=\sum^{E_{m}\ne E_{n}}_{n,m}
\big(\bra{m} B^{\dagger} \ket{n} \bra{n} A\ket{m}\big)^{*}
\frac{\rho_{m}-\rho_{n}}{E_{n}-E_{m}}\\
%&=\sum^{E_{m}\ne E_{n}}_{n,m}
%\bra{n} A\ket{m}^{*} \bra{m} B^{\dagger} \ket{n}^{*}
%\frac{\rho_{m}-\rho_{n}}{E_{n}-E_{m}}\\
&=\sum^{E_{m}\ne E_{n}}_{n,m}
\bra{m} A^{\dagger}\ket{n} \bra{n} B \ket{m}
\frac{\rho_{m}-\rho_{n}}{E_{n}-E_{m}}\;,\\
&=\big( A|B\big)\;.\qquad\checkmark
\end{align*}
\item Linearity $\big( A|\alpha B_{1} + \beta B_{2}\big) =\alpha \big( A| B_{1}\big) + \beta \big(A|B_{2}\big) $, which follows from the linearity of the matrix elements
$\bra{n} B\ket{m}$.
\item $\big( A|A \big)\ge 0$, since 
%
\begin{align*}
\big( A|A \big) &= \sum^{E_{m}\ne E_{n}}_{n,m}
\underbrace{
\abs{\bra{n} A^\dagger \ket{m}\big)}^{2}
}_{\color{blue} \ge 0 }
\underbrace{
\frac{\rho_{m}-\rho_{n}}{E_{n}-E_{m}}
}_{\color{blue} \ge 0 }\;.
\end{align*}
%
\item From  $A=0$ follows clearly $\big( A|A \big)= 0$, but not conversely  
%
\end{itemize}
So we can conclude that $( A|B )$ is a semi-definite scalar product, for which the
Schwarz inequality applies.
In addition, the product has the special feature
%
\begin{align}\label{eq:schwarz}
\big( H|A \big)  &= 0\;,\qquad \forall \text{operators } A\;.
\end{align}
%
This is easily seen, since $\bra m H^{\dagger} \ket{n} = E_{n} \delta_{mn}$; but $m=n$
is excluded from he sum.



Next we exploit it for the prove of the Bogoliubov inequality. To this end, we first use
$B=[C^{\dagger},H]$ in the Schwarz relation. For the lhs we need $(A|B)$
%
\begin{align*}
\big( A|B \big) &=\big( A|[C^{\dagger},H]\big)\\
&=\sum^{E_{m}\ne E_{n}}_{mn}
\bra{m} A^{\dagger}\ket{n}\;\bra{n}\big[ C^{\dagger},H \big]\ket{m}
\frac{\rho_{m}-\rho_{n}}{E_{n}-E_{m}}\\
&=\sum^{E_{m}\ne E_{n}}_{mn}
\bra{m} A^{\dagger}\ket{n}\;\bra{n}C^{\dagger}\ket{m}\big( E_{m}-E_{n} \big)
\frac{\rho_{n}-\rho_{m}}{E_{m}-E_{n}}\\
&=\sum^{E_{m}\ne E_{n}}_{mn}
\bra{m} A^{\dagger}\ket{n}\;\bra{n}C^{\dagger}\ket{m}
\big(\rho_{n}-\rho_{m}\big)\;.
\end{align*}
%
Now we can omit the restriction in the sum, since $\big(\rho_{n}-\rho_{m}\big)$ 
is zero for $n=m$, anyways. Then
\begin{align*}
\big( A|B \big) &=\sum_{mn}
\rho_{n} \bra{n}C^{\dagger}\ket{m} \;\bra{m} A^{\dagger}\ket{n}\;
-
\sum_{mn}\rho_{m}
\bra{m} A^{\dagger}\ket{n}\;\bra{n}C^{\dagger}\ket{m}
\;.
\end{align*}
%
Renaming $n\leftrightarrow m$ in the last term 
eventually leads to
\begin{align}\label{eq:schwarz:2}
\big( A|B \big) &=\avg{ \big[C^{\dagger}, A^{\dagger}\big]_{-}}\;
\end{align}
If we also choose  $A=B=[C^{\dagger},H]_{-}$ in \eq{eq:schwarz:2}, we obtain
%
\begin{align}\label{eq:schwarz:3}
\big( B|B  \big) &= 
\avg{ \big[C^{\dagger}, \big[ H,C \big]_{-}\big]_{-}}
=\avg{ \big[\big[ C,H \big],C^{\dagger}\big]_{-}}\ge 0\;.
\end{align}
%
For the next steps we use ($\rho_{m}= e^{-\beta E_{m}}/Z$)
%
\begin{align*}
\rho_{m} &> \rho_{n} \;\quad \text{if}\quad E_{n} > E_{m}\;.
\end{align*}
%
Hence, for any $E_{n}\ne E_{m}$
%
\begin{align}\label{eq:bogoliubov1}
0 &< \frac{\rho_{m}-\rho_{n}}{E_{n}-E_{m}} = \frac{\rho_{m}+\rho_{n}}{E_{n}-E_{m}}\cdot 
\frac{\rho_{m}-\rho_{n}}{\rho_{m}+\rho_{n}}\;.
\end{align}
%
The second factor can be modified 
%
\begin{align*}
\frac{\rho_{m} - \rho_{n}}{\rho_{m}+\rho_{n}} &= 
\frac{e^{-\beta E_{m}}-e^{-\beta E_{m}}}{e^{-\beta E_{n}}+e^{-\beta E_{m}}}\\
&=\frac{e^{-\frac{\beta}{2} E_{m}}e^{-\frac{\beta}{2} E_{n}}}{e^{-\frac{\beta}{2} E_{m}}e^{-\frac{\beta}{2} E_{n}}}\;
\frac{
e^{-\frac{\beta}{2} \big(E_{m}-E_{n}\big)}-e^{-\frac{\beta}{2} \big(E_{n}-E_{m}\big)}
}{
e^{-\frac{\beta}{2} \big(E_{m}-E_{n}\big)}+e^{-\frac{\beta}{2} \big(E_{n}-E_{m}\big)}
}\\
&=\tanh\bigg( \frac{\beta}{2}\big(E_{n}-E_{m}\big) \bigg)
\end{align*}
%
Then \eq{eq:bogoliubov1} yields
%
\begin{align*}
0<&\frac{\rho_{m}-\rho_{n}}{E_{n}-E_{m}}  = \frac{\beta}{2}\;\big(\rho_{m}+\rho_{n}\big) \frac{\tanh\big(\frac{ \beta}{2}(E_{n}-E_{m} \big)}{\frac{\beta }{2} \big( E_{n}-E_{m} \big)}\;.
\end{align*}
%
Since $\tanh(x)/x < 1$, we find
\begin{align*}
0 &< \frac{\rho_{m}-\rho_{n}}{E_{n}-E_{m}} 
< \frac{\beta}{2}\;\big(\rho_{m}+\rho_{n}\big) \;.
\end{align*}
From this relation we obtain
%
\begin{align*}
\big( A|A \big) &= \sum_{mn}^{E_{m}\ne E_{n}}
\bra{n} A^{\dagger} \ket{m}\bra{m} A\ket{n}\frac{\rho_{m}-\rho_{n}}{E_{n}-E_{m}}
&< \frac{\beta}{2}\sum_{mn}^{E_{m}\ne E_{n}}
\bra{n} A^{\dagger} \ket{m}\bra{m} A\ket{n}\;\big(\rho_{m}+\rho_{n}\big)\;.
\end{align*}
%
The omitted terms for $E_{n}=E_{m}$ are positive, so including them does not violate  the inequality
%
\begin{align}\label{eq:schwarz:4}
\big( A|A \big) &< \frac{\beta}{2}\sum_{mn}
\bra{n} A^{\dagger} \ket{m}\bra{m} A\ket{n}\;\big(\rho_{m}+\rho_{n}\big)
=\frac{ \beta}{2} \langle \big[A^{\dagger},A \big]_{+}\rangle
\end{align}
%

Now we return to the Schwarz inequality in \eq{eq:schwarz:a} 
\begin{align}
\abs{\big( A|B \big)}^{2} \le \big( A|A \big)\big( B|B \big)\;,
\end{align}
and insert piece by piece the relations we just derived for the various terms. Beginning with \eq{eq:schwarz:2} and \eq{eq:schwarz:3} we get
\begin{align*}
\big|\avg{\big[ C,A \big]}\big|^{2}
 \le \big( A|A \big)\avg{ \big[\big[ C,H \big],C^{\dagger}\big]_{-}}\;,
\end{align*}

Along with \eq{eq:schwarz:4} we finally obtain
\begin{align}
\big|\avg{\big[ C,A \big]}\big|^{2}
 \le \frac{\beta}{2}\langle \big[A^{\dagger},A \big]_{+}\rangle \avg{ \big[\big[ C,H \big],C^{\dagger}\big]_{-}}\;,
\end{align}
which proves the Bogoliubov inequality.



\subsection{Proof used for the Mermin-Wagner theorem \label{app:prove:Bogoliubov:3}}

Here we prove the relation
%
\begin{align}\label{eq:tobeproven}
 \avg{\bigg[ \big[C,H  \big]_{-},C^{\dagger} \bigg]_{-}}&\le 
4 \hbar^{2} N\bigg( \abs{b {\cal S}_{\vv Q}(T,b)} +  \vv k^{2} X \bigg)\;.
\end{align}
%
We recall that 
%
\begin{align*}
C&=S^{+}_{\vv k} = \sum_{l} e^{i \vv k\vv R_{l}} S^{+}_{l}\;.
\end{align*}
Hence
%
%
We begin with $[S_{k}^{+},H]_{-}$
%
\begin{align*}
[C,H_{H}]_{-} &= [S^{+}_{\vv k},H_{H}]_{-} \\
&=-\sum_{l}e^{i\vv R_{l}\vv k} \sum_{jj'}J_{jj'}\bigg(\big[ S_{l}^{+},S^{+}_{j} S^{-}_{j'}\big]
+ \big[S_{l}^{+},S^{z}_{j} S^{z}_{j'}\big]\bigg)\\
&=-\sum_{l}e^{i\vv R_{l}\vv k} \sum_{jj'}J_{jj'}\bigg(S^{+}_{j}\big[ S_{l}^{+}, S^{-}_{j'}\big]
+ S^{z}_{j}\big[S_{l}^{+}, S^{z}_{j'}\big]
+ \big[S_{l}^{+},S^{z}_{j}\big] S^{z}_{j'}
\bigg)\\
&=-\hbar \sum_{l}e^{i\vv R_{l}\vv k} \sum_{jj'}J_{jj'}\bigg(2 S^{+}_{j}
\delta_{lj'} S^{z}_{l}
- S^{z}_{j} \delta_{lj'} S_{l}^{+}
- \delta_{lj}S_{l}^{+}S^{z}_{j'}
\bigg)\\
&=-\hbar \sum_{l}e^{i\vv R_{l}\vv k} \sum_{j}J_{jl} \bigg(2 S^{+}_{j}
 S^{z}_{l}
-S^{z}_{j} S_{l}^{+}
-  S_{l}^{+}S^{z}_{j}\bigg)\\
&=-2\hbar \sum_{l}e^{i\vv R_{l}\vv k} \sum_{j}J_{jl} 
\bigg(S^{+}_{j} S^{z}_{l}- S_{l}^{+}S^{z}_{j}\bigg)\;.
\end{align*}
%
We have used the symmetry $J_{jj'}=J_{j'j}$ and $S^{z}_{j} S_{l}^{+}=
S_{l}^{+} S^{z}_{j} +[S^{z}_{j}S_{l}^{+}] $. The commutator is proportional to $\delta_{jl}$ and hence to $J_{jj}$, which is zero. Meanwhile we have
\begin{align*}
[C,H_{H}]_{-}
&=-2\hbar \sum_{lj} \big(e^{i\vv R_{l}\vv k}  - e^{i\vv R_{j}\vv k} \big)\;J_{jl}
\;S^{+}_{j} S^{z}_{l}\;.
\end{align*}
We continue with the second commutator 
%
\begin{align*}
\big[[C,H_{H}]_{-},C^{\dagger}\big]_{-}&=
-2\hbar \sum_{lj} \big(e^{i\vv R_{l}\vv k}  - e^{i\vv R_{j}\vv k} \big)\;J_{jl}
\;\sum_{l'} e^{-i\vv k\vv R_{l'}}\big[S^{+}_{j} S^{z}_{l},S^{-}_{l'}\big]\;.
\end{align*}
%
With
%
\begin{align*}
\big[S^{+}_{j} S^{z}_{l},S^{-}_{l'}\big] &=
S^{+}_{j}  \big[S^{z}_{l},S^{-}_{l'}\big]
+\big[S^{+}_{j},S^{-}_{l'}\big] S^{z}_{l}\\
&=-\delta_{ll'}\hbar S^{+}_{j}  S^{-}_{l}
+2 \hbar \delta_{l'j} S^{z}_{j}S^{z}_{l}
\end{align*}
%
we obtain
\begin{align*}
\big[[C,H_{H}]_{-},C^{\dagger}\big]_{-}&=
-2\hbar^{2} \sum_{lj} \big(e^{i\vv R_{l}\vv k}  - e^{i\vv R_{j}\vv k} \big)\;J_{jl}
\bigg(- e^{-i\vv k\vv R_{l}}S^{+}_{j}  S^{-}_{l}
+2 e^{-i\vv k\vv R_{j}} S^{z}_{j}S^{z}_{l}
\bigg)\\
&=
2\hbar^{2} \sum_{lj}\;J_{jl}
\bigg(  \big(1 - e^{i(\vv R_{j}-\vv R_{l})\vv k} \big)S^{+}_{j}  S^{-}_{l}
-2\big( e^{-i(\vv R_{j}-\vv R_{l})\vv k} -1\big) S^{z}_{j}S^{z}_{l}
\bigg)\\
&=
2\hbar^{2} \sum_{lj}\;J_{jl}
  \bigg(1 - e^{i(\vv R_{j}-\vv R_{l})\vv k} \bigg)
\bigg(S^{+}_{j}  S^{-}_{l} + 2 S^{z}_{j}S^{z}_{l}\bigg)\;.
\end{align*}
In the last step we have substituted $j\leftrightarrow l$ and used the symmetry of $J_{jl}$.
Finally, we repeat the calculation for the field term
%
\begin{align*}
[C,H_{b}]_{-} &= [S^{+}_{\vv k},H_{b}]_{-} 
= b \sum_{ll'}e^{i\vv R_{l}\vv k} e^{i\vv R_{l'}\vv Q}\;\underbrace{
\big[ S_{l}^{+},S^{z}_{l'}\big]
}_{\color{blue} = -\delta_{ll'}\hbar S^{+}_{l}}\\
&=b\hbar \sum_{l}  e^{i\vv R_{l}(\vv k +\vv Q)}	 S^{+}_{l}\;.
\end{align*}
%
Then
\begin{align*}
\big[[C,H_{b}]_{-},C^{\dagger}\big] 
&=b\hbar \sum_{l}  e^{i\vv R_{l}(\vv k +\vv Q)}	\sum_{l'}e^{-i \vv k\vv R_{l'}} 
\underbrace{
\big[S^{+}_{l},S^{-}_{l'}\big]
}_{\color{blue} = \delta_{ll'} 2\hbar S^{z}_{l}}\\
&=2 b\hbar^{2} \sum_{l}  e^{i\vv R_{l} \vv Q}	 S^{z}_{l}\\
&=2 b\hbar^{2}  S^{z}_{\vv Q} \\
&= 2 b N \hbar^{2}{\cal S}_{\vv Q}(T,b)\;.
\end{align*}
Hence for $C=C_{\vv k}$ we have
\begin{align*}
\avg{\big[[C_{\vv k},H]_{-},C^{\dagger}_{\vv k}\big] }
&=2 bN\hbar^{2}  {\cal S}_{Q}(T,b)\\
&+\quad 2\hbar^{2} \sum_{lj}\;J_{jl}
  \bigg(1 - e^{i(\vv R_{j}-\vv R_{l})\vv k} \bigg)
\bigg(\underbrace{
\langle S^{+}_{j}  S^{-}_{l}\rangle + \langle S^{z}_{j}S^{z}_{l}\rangle
}_{\color{blue} = \avg{\vv S_{j}\cdot \vv S_{l}}}
+\langle S^{z}_{j}S^{z}_{l}\rangle\bigg)\\
&=2 b N\hbar^{2}  {\cal S}_{Q}(T,b)\\
&+\quad 2\hbar^{2} \sum_{lj}\;J_{jl}
  \bigg(1 - e^{i(\vv R_{j}-\vv R_{l})\vv k} \bigg)
\bigg( \avg{\vv S_{j}\cdot \vv S_{l}}
+\avg{S^{z}_{j}S^{z}_{l}}\bigg)\;.
\end{align*}
%
From \eq{eq:schwarz:3} we know  that $\big[[C,H_{b}]_{-},C^{\dagger}\big] $ originates from the scalar product  $(B|B)$. It is therefore a real and non-negative function for all $\vv k$. Then
%
\begin{align*}
\avg{\big[[C_{\vv k},H]_{-},C^{\dagger}_{\vv k}\big] } &\le 
\avg{\big[[C_{\vv k},H]_{-},C^{\dagger}_{\vv k}\big] }+
\avg{\big[[C_{-\vv k},H]_{-},C^{\dagger}_{-\vv k}\big] }
\\
\avg{\big[[C_{\vv k},H]_{-},C^{\dagger}_{\vv k}\big] }  &\le 4 b\hbar^{2} N {\cal S}_{Q}(T,b)\\
&+\quad 4\hbar^{2} \sum_{lj}\;J_{jl}
  \bigg(1 - \cos\big(\vv R_{j}-\vv R_{l})\vv k\big) \bigg)
\bigg( \avg{\vv S_{j}\cdot \vv S_{l}}
+\avg{S^{z}_{j}S^{z}_{l}}\bigg)\\
&\le 4 N\hbar^{2}  \abs{b{\cal S}_{Q}(T,b)}\\
&+\quad 4\hbar^{2} \sum_{lj}\;
\underbrace{
\abs{J_{jl}}
  \bigg(1 - \cos\big(\vv R_{j}-\vv R_{l})\vv k\big) \bigg)
}_{\color{blue}  \ge 0}
\bigg( \abs{\avg{\vv S_{j}\cdot \vv S_{l}}}
+\abs{\avg{S^{z}_{j}S^{z}_{l}}}\bigg)\;.
\end{align*}
%
Finally, we can use the Schwarz inequality for the scalar products $\avg{\vv S_{j} \cdot \vv S	_{l}}$
%
\begin{align*}
\abs{\avg{\vv S_{j} \cdot \vv S	_{l}}}^{2} &\le 
\avg{\vv S_{j}^{2}}
\avg{\vv S_{l}^{2} }
 =\bigg(\hbar^{2} S(S+1)\bigg)^{2}\\
 \abs{\avg{\vv S_{j} \cdot \vv S	_{l}}}&\le \hbar^{2}S(S+1)\;.
\end{align*}
%
Moreover, $\abs{\avg{S^{z}_{j}S^{z}_{l}}}\le \abs{\avg{S_{j}S_{l}}}\le \hbar^{2}S(S+1)$.
In total we have
%
\begin{align*}
\avg{\big[[C,H]_{-},C^{\dagger}\big] } &\le 
4 N \hbar^{2}  \bigg(\abs{b{\cal S}_{Q}(T,b)}
+ 2S(S+1)\hbar^{2}\;
\frac{1}{N}\sum_{jl}\abs{J_{jl}}
  \bigg(1 - \cos\big(\vv R_{j}-\vv R_{l})\vv k\big) \bigg)
\bigg)
\end{align*}
%
Moreover we use
%
\begin{align*}
1-\cos(x) \le \frac{x^{2}}{2}\;,
\end{align*}
%
and obtain
\begin{align*}
\avg{\big[[C,H]_{-},C^{\dagger}\big] } &\le 
4 N \hbar^{2}  \bigg(\abs{b{\cal S}_{Q}(T,b)}
+ \;\vv k^{2}
\underbrace{
\frac{S(S+1)\hbar^{2}}{N}\sum_{jl}\abs{J_{jl}}\abs{\vv R_{j}-\vv R_{l}}^{2}
}_{\color{blue} = X \text{with } 0\le X < \infty}
   \bigg)
\bigg)
\end{align*}
Which proves \eq{eq:tobeproven}.
